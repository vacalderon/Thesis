\chapter{Introduction}
\label{chap-one}

Structures are designed assuming their original condition remains intact through their service life. However, as structures age, they suffer various forms of degradation. In addition, they may be subjected to multiple discrete seismic events. Both of these items may impact  structural performance. Consider an RC column and the limit state corresponding to bar buckling. The limit state displacement ($\Delta$) to achieve bar buckling for a pristine column is different from a column subjected to corrosion. Similarly, multiple small seismic events may predispose the column to suffer longitudinal bar buckling for a lower level of seismic intensity. It is the goal of this research to incorporate both of these effects into the definition of performance limit states. This can potentially serve two purposes: a) for existing structures, assessment can consider likely current conditions, and b) for new structures, changes could be proposed to the initial design that could mitigate the effect of future condition degradation.

Structures subjected to multiple events and aging conditions should be considered in Performance-Based Earthquake Engineering (PBEE). Recent earthquakes sequences such as the Christchurch 2010, Umbria-Marche Earthquake 1997 and more recently the Puerto Rico Earthquakes 2020, have shown that structures after sustaining damage during a mainshock have then collapsed or sustained increased damage after being subjected to a large magnitude aftershock\cite{Amato1998}\cite{Bradley}\cite{Miranda2020}. Researchers have used the Park and Ang damage index (DI) to quantify damage, which is expressed as a two-term expression. The first term relates to the maximum displacement and the second term relates to the inelastic energy dissipation \cite{Young-JiPark1985}.The second term is associated with the inelastic cyclic behavior of structural components.In addition, the calibration factor is very small and contributes little to the damage index. If the damage index renders the inelastic cyclic behavior as negligible, it cannot accurately represent damage. Further, this damage index uses calibrated data to determine the strength degradation parameter that has a degree of arbitrariness, which is undesirable \cite{Williams1995}. In addition to the Park and Ang damage index, other measures of damage such as drift ratio based limit states have been incorporated into the PEER Performance-Based Design Probabilistic Framework\cite{Padgett2007}\cite{Ghosh2015}\cite{Shekhar2018}. The majority of these studies show that there is an increase in the probability of damage or even collapse of a structure due to repeated loading or aging conditions, such as high corrosion levels (CL). However, these results are based on the limitations presented by these damage measures \cite{Shekhar2018}. A more deterministic approach to measure damage is to use strain limits as indicators of damage for RC bridge columns, where concrete compressive and reinforcing steel tensile strain limits are used as the damage measure \cite{Goodnight2016}. These strain limits have been correlated to observed damage in large scale column tests. Therefore, we believe that our research will provide a realistic measure of the increase in damage for different limit states due to aging conditions and multiple earthquake loadings. 

In addition, structures can have an existing condition such as corrosion that further deteriorates the performance of the structure. Corrosion is one of the aging conditions that more significantly deteriorates the seismic response of a structure. Thus, it is important to determine the limit states of corroded reinforcing steel. Currently, the literature has developed expressions that correlate the level of corrosion to the decrease in strength of the reinforcing steel\cite{Yuan2017a}\cite{Du2005}. However, these studies have utilized an accelerated corrosion process that does not consider the protective film that is developed on the reinforcing steel surface when it is embedded in concrete, a process known as passivation of the reinforcing steel \cite{Mehta2014}\cite{Ghods2009}. For corrosion to occur, the protective film on the reinforcing steel bar must occur, this process is known as  depassivation. Depassivation of the reinforcing steel greatly affects the behavior of reinforcing steel and can significantly modify the measured properties of the corroded reinforcing steel. Furthermore, no study has presented performance limit states on corroded reinforcement. Therefore, this research aims to close this gap by performing an experimental campaign. This experimental campaign consists of a series of tension tests and buckled bar tension tests to help define the performance limit states of corroded reinforcement. These results will then inform the computational model.

Moreover, there is a high likelihood for a structure in a high seismic region to be subjected to mainshock-aftershock sequence during its service life. Therefore, it is important to consider the effects of mainshock-aftershock sequences. A series of condition-dependent nonlinear time history analyses are performed on a cantilever RC bridge column. The analysis applies a series of MS-AS sequence for different ages of the structure. The material properties of the structure are changed as a function of the aging conditions (e.g. corrosion). At the end of each analysis, the main variables of the study are the limit state that was reached, the controlling mode of response (flexural or shear controlled), the equivalent viscous damping, and the accumulated deformations.

In summary, this research aims to consider the likely future condition (considering deterioration mechanism and multiple earthquake loadings) of a structure in defining strain-based performance limit states.

\section{Scope and layout}
This research proposal describes the main components and objectives for the graduate studies of the author of this document. Chapter 2 contains the literature review which summarizes the state of the art in damage measurements and performance-based design framework. Chapter 3 covers the experimental campaign in corroded reinforcing steel. Chapter 4 summarizes the experimental and analytical procedures relevant to this study. Chapter 5 details the processing of current results. 