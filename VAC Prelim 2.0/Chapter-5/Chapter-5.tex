\chapter{Expected findings and schedule of work}
Developing a design tool that provides resilient bridges. Therefore, this research will develop a displacement based design tool, that achieves that goal. In this chapter introduce the overall design goal. Then the direct displacement based design (DDBD) is presented and a proposed modifications to this design methodology are shown. Finally a plan of work is presented.

\section{Condition dependent performance based design concept}

Current design methodologies in earthquake engineering follow the performance based dseign approach. This methodology has allowed engineers and stke holders of infrastructure projects to design and build structures that will achieve a performance goal during the service life of the structure. Implied in this approach is the assumption that the properties of the structure will remain in pristine conditions. It is clear from the existing structures that structures age, and from the information presented in the previous chapters, it is possible to account for this aging. Figure X.X schematically shows what would happen to structures designed to pristine conditions as it ages. This design would render a structure close to damage limit states if left alone, sooner than the assumed service life. On the other hand our design approach while it won't get rid of aging, it will allow for structures to remain well above a prescribed limit state before the structure reaches the end of it's service life. 

\section{Condition dependent displacement based design (CD-DDBD)}
\subsection{Overview of displacement based design}

Displacement based design consists in ...

\subsection{Proposed design methodology}

Recent studies conducted in corroded RC columns have shown that the performance of these systems is greatly deterred by corrosion. These studies have shown that corroded structures have a lower strength, and displacement capacity. In addition, it appears that their hysteresis area is lower compared to a pristine condition RC column. This implies that their hysteretic damping component follows the same trend. Using the Jacobsen approach it can be seen that as the corrosion level increases, the hysteretic damping decreases too. Similarly as the axial load ration (ALR) in the columns increases, the hysteretic damping decreases. The scopes of these studies is limited, since the detailing used is not what is prescribed in modern structures, and the accelerated corrosion process used high levels of current density, further affecting the measured results. 

In this study we are proposing to evaluate changes in the hysteretic damping used in corroded RC structures, as well as changes in the design displacement since there is limit states such as buckled bar can occur at a sooner displacement compared to a pristine condition design. Furthermore in some cases the design could be controlled by shear due to the deterioration of the entire system.
