\chapter{Expected findings and schedule of work}
Developing a design procedure that provides resilient bridges is one of the main goals of this study.This research will develop a displacement based design tool, that achieves that goal. This chapter firts introduces the overall design concept; then the direct displacement based design (DDBD) is reviewed and proposed modifications to this design methodology, based on corroded RC physical tests results, are shown. Finally a schedule of work is presented.

\section{Condition dependent performance based design concept}

Current design methodologies in earthquake engineering follow the performance based design philosophy. This methodology has allowed engineers and stake holders of infrastructure projects to design and build structures that will achieve a performance goal during the service life of the structure. Implied in this methodology is the assumption that the properties of the structure will remain unchanged, specifically, that the structure will maintain its pristine conditions properties. We know by observing existing structures in our environment that structures age, and from the information presented in the previous chapters, it is possible to account for this aging when designing resilient structures.  Figure \ref{fig:Concept_CD-DDBD}schematically shows what would happen to structures designed to pristine conditions as they age. Essentially, current designs would render a structure close to damage limit states if left alone, sooner than the assumed service life. On the other hand, while our design approach won’t eliminate aging, it aims to allow structures to remain well above a prescribed limit state before the structures reach the end of their service life. 

\begin{figure}[htbp]
	\centering
	\includegraphics[width=0.90\textwidth]{VAC Prelim 2.0/Chapter-5/figs/CD_DDBD_Concept.png}
	\caption{Schematic concept of condition dependent performance based design and traditional designs}
	\label{fig:Concept_CD-DDBD}
\end{figure}

One of the design procedures in performance  based design is the direct displacement based design methodology. This research will modify components of the DDBD procedure to account for aging of the structures. The following sections present a summary on the DDBD methodology, as well as proposed components of the DDBD methodology that could be modified to incorporate our findings.

\section{Condition dependent direct displacement based design (CD-DDBD)}
\subsection{Overview of direct displacement based design}

While direct displacement based design (DDBD) is a well-known procedure (see \cite{Priestley2007}), an overview of the methodology is presented here for convenience for a SDOF system. DDBD consists of five fundamental steps. 1) Characterize the limit states of the structure. These limit states correspond to structural damage or prescribed maximum drifts. For instance, in corroded RC circular columns, buckling of the corroded reinforcement is a limit state that precedes the strength degradation of the system. These limit states are then used to calculate the target displacements ($\Delta_{d}$) for the design. The goal of DDBD it to ensure that the structure reaches a desired performance by reaching the target displacement for a given displacement spectra.  2)  Calculate the equivalent viscous damping ($\xi_e$) using equations developed for different structural systems. Equivalent viscous damping depends on the expected ductility at the target displacement, and the force displacement hysteresis shape of the structural system ($\xi_{hyst}$). For example a modernly detailed RC structure will have a "fatter" hysteresis shape than an older structure. 3)  Determine the effective period ($T_e$) of the structure by entering with the target displacement ($\Delta_d$) and intersecting the design displacement spectra at the level of damping found in step (2) . The effective period is  the period of the structure at peak response. 4) Calculate the effective stiffness using the dynamic properties of the equivalent structure, plugging in the values in $K_e=\frac{4\pi^{2}m_{e}}{T_{e}}$. 5) Finally, calculate the base shear using the stiffness equation, $V_{be}=K_{e}\Delta_{d}$.

\begin{figure}[htbp]
	\centering
	\includegraphics[width=0.75\textwidth]{VAC Prelim 2.0/Chapter-5/figs/DDBD.png}
	\caption{Direct displacement based design method summary \cite{Priestley2007}}
	\label{fig:DDBD_sum}
\end{figure}
\newpage

\subsection{Proposed design methodology}

Recent studies conducted in corroded RC columns have shown that the performance of these systems greatly decreased corrosion. These studies have shown that corroded structures have a lower strength and displacement capacity, as shown below. We hypothesize that the hysteresis area of the force displacement is lower in the corroded RC structure when compared to its pristine counterpart. For the DDBD procedure, this implies that the equivalent damping component follows the same trend. An analysis of the results of corroded structure is further expanded in this section. 

\textbf{Jacobsen damping in corroded RC structure physical tests.}

Figure \ref{fig:MedaJacobsen} shows the force-displacement relationship between corroded, and pristine condition RC structures, from a study performed by Meda et al  \cite{Meda2014}. Jacobsen damping is used to quantify the hysteretic damping, as shown in \eref{eq:JacobsenEquation}. For these structures, $\xi_{CL=0}=31.4\%$, and for the corroded structure $\xi_{CL=0}=24.2\%$, which corresponds to a reduction of $23\%$. A similar trend is observed when performing the same analysis on the results of a comparable  study from Ma et al \cite{Ma2012}.

\begin{equation}
    \xi=\frac{A_h}{2*\pi*F_m*\Delta_m}
    \label{eq:JacobsenEquation}
\end{equation}

Figure \ref{fig:JacobsenResults} shows a plot of the Jacobsen damping for these tests in function of the corrosion level (CL), and the level of axial load ratio ($ALR=\frac{P}{A_{g}f'_{c}}$) . Our analysis of these studies shows that as the corrosion level and the axial load ratio in the columns increases, the hysteretic damping decreases. The scope of their studies was limited due to the non-modern detailing used and due to the fact that their accelerated corrosion process used high levels of current density. However, they provide insight to the consequences of corrosion in hysteretic damping. 

\begin{figure}[htbp]
	\centering
    \includegraphics[width=0.75\textwidth]{VAC Prelim 2.0/Chapter-5/figs/Meda_HystereticArea_01.png}
	\caption{Hysteretic energy dissipation in pristine and corroded RC from Meda et al \cite{Meda2014}}
	\label{fig:MedaJacobsen}
\end{figure}

\begin{figure}[htbp]
	\centering
    \includegraphics[width=0.75\textwidth]{VAC Prelim 2.0/Chapter-5/figs/Ma_HystereticArea_01.png}
	\caption{Hysteretic energy dissipation in pristine and corroded RC from Ma et al \cite{Meda2014}}
	\label{fig:MaJacobsen}
\end{figure}

\begin{figure}[htbp]
	\centering
    \includegraphics[width=0.75\textwidth]{VAC Prelim 2.0/Chapter-5/figs/HystereticDampingLitResults.png}
	\caption{Jacobsen hysteretic damping from Ma et al \cite{Ma2012}, and Meda et al \cite{Meda2014}}
	\label{fig:JacobsenResults}
\end{figure}

\textbf{Proposed changes DDBD}

Corrosion greatly impacts the hysteretic damping of corroded RC structures and changes in the limit states of corroded rebars, as a result of the experimental program outcomes. Furthermore, in some cases the design could be controlled by shear due to the deterioration of the entire system. Therefore, we propose including the effects of aging in two different inputs in DDBD by changing 1) the design displacement, which is calculated using the limit states that relate to corrosion, and 2) the equations used to determine the equivalent damping, which can be modified with a factor that relates to corrosion.
\newpage
\section{Schedule of work}