%% ------------------------------ Abstract ---------------------------------- %%
\begin{abstract}

Structures located in seismic prone regions are often subjected to multiple earthquakes. Multiple earthquakes can accumulate damage resulting in a deterioration of the seismic performance of a structure. In addition, aging of the structure can lead to corrosion that further propagates the deterioration. Past research has shown that the Park and Ang damage index (DI)(a demand parameter that quantifies damage) increases as damage aging conditions in the structures worsen or multiple seismic events are included in the analysis. This increases the probability of a structure to collapse. This research proposes to study the effects of multiple earthquakes and damage accumulation in RC structures by developing strain limit states fragility functions for different aging conditions. To achieve this it is also important to develop limit states that represent corroded reinforcing steel. A method to perform accelerated corrosion in passivated reinforcing steel is proposed. These corroded rebars are then subjected to tension tests and buckled bar tension tests, which will later be used to define service limit state and damage control limit state. To show the relevance of this study a framework that incorporates corrosion models into a nonlinear time history analysis (NLTHA)   is developed. A series of SDOF cantilever columns are subjected to a sweep of earthquakes.  These preliminary results show that there is an increase in the probability of reaching a limit state when corrosion level increases. The results also show the dispersion of results by using  PGA as the impact measure (IM), indicating the need for a better intensity measure. The results of this research will (1) develop fragility curves that consider strain limit states to measure damage while incorporating different aging conditions, (2) establish limit states for corroded rebars, (3) inform the research community on the necessary methodology to accurately model corrosion for material testing and large scale testing of corroded reinforced members (4) consider the effects of multiple earthquakes for mainshock sequences and mainshock-aftershock sequence (5) incorporate the results into the direct displacement-based design methodology.


\end{abstract}


%% ---------------------------- Copyright page ------------------------------ %%
%% Comment the next line if you don't want the copyright page included.
\makecopyrightpage

%% -------------------------------- Title page ------------------------------ %%
\maketitlepage

%%% -------------------------------- Dedication ------------------------------ %%
%\begin{dedication}
% \centering To my parents. To God. To Aubrey.
%\end{dedication}
%
%%% -------------------------------- Biography ------------------------------- %%
%\begin{biography}
%The author was born in land far away where the earth rocks like a hammock. That land name is El Salvador. \ldots
%\end{biography}
%
%%% ----------------------------- Acknowledgements --------------------------- %%
%\begin{acknowledgements}
%I would like to thank Dr. Kowalsky for his help. The Staff and Students at The Constructed Facilities Lab of NC State. The Alaska Department of Transportation. \ldots
%\end{acknowledgements}


\thesistableofcontents

\thesislistoftables

\thesislistoffigures
