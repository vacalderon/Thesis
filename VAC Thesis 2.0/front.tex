% ------------------------------ Abstract ---------------------------------- %
\begin{abstract}

Structures located in seismic regions are often subjected to multiple earthquakes. Multiple earthquakes can accumulate damage resulting in a deterioration of the seismic performance of a structure. In addition, aging of the structure can lead to corrosion that further propagates the deterioration. Past research has shown that the Park and Ang damage index (DI)(a demand parameter that quantifies damage) increases as damage aging conditions in the structures worsen or multiple seismic events are included in the analysis. This increases the probability of structures to collapse. This research proposes to study the effects of multiple earthquakes and damage accumulation in RC structures by developing strain based fragility functions for different aging conditions. To achieve this it is also important to develop limit states that represent corroded reinforcing steel. Therefore, a method to perform accelerated corrosion in passivated reinforcing steel rebars is proposed. These corroded rebars are then subjected to tension tests and buckled bar tension tests, which will be used to define service and damage control limit states. To show the relevance of this study a framework that incorporates corrosion models into a nonlinear time history analysis (NLTHA) is developed. A series of SDOF cantilever columns are subjected to a sweep of earthquakes.  Preliminary results show that there is an increase in the probability of reaching a limit state when corrosion level increases. The results also show large dispersion of results when using  PGA as the intensity measure (IM), indicating the need for a better intensity measure. The results of this research will (1) develop fragility curves that consider strain limit states to measure damage while incorporating different aging conditions, (2) establish limit states for corroded rebars, (3) inform the research community on the necessary methodology to accurately model corrosion for material testing and large scale testing of corroded reinforced members (4) consider the effects of multiple earthquakes with mainshock-aftershock sequences (5) incorporate the results into the direct displacement-based design methodology.


\end{abstract}


%% ---------------------------- Copyright page ------------------------------ %%
%% Comment the next line if you don't want the copyright page included.
\makecopyrightpage

%% -------------------------------- Title page ------------------------------ %%
\maketitlepage

%%% -------------------------------- Dedication ------------------------------ %%
\begin{dedication}
\centering
To God, "How much better it is to get wisdom than gold! And to get understanding is to be chosen above silver" (Prov 16:16). 
 \newline To my parents Victor M. Calderon and Roxana P. de Calderon, for all their love and support through the years and through the miles. 
 \newline To my wife Aubrey, "If it takes doing a Ph.D. for us to meet it will have been worth it". Thank you for your love, support, patience and for filling my days with joy.
\end{dedication}
%
%%% -------------------------------- Biography ------------------------------- %%
\begin{biography}
I was born in San Salvador, El Salvador. I attended the Universidad Centroamericana José Simeón Cañas, where I graduated as Civil Engineer in 2012. Subsequently, I started working for the structural engineering consulting firm JEP Ingenieros y Arquitectos, where I gained experience designing Commercial buildings, schools and residential buildings. While still being a part of JEP, I pursued my Master in Civil Engineering at NC State from 2014-2016. In 2016, I started to work for a hydropower company, where I was part of the team that designed the facilities for a large dam project. In the Spring of 2018, I decided to return to NC State for the PhD program. In 2021, I became a registered Professional Engineer in the State of North Carolina. Going forward, I plan to continue my career as a structural engineer in a consulting firm.
\end{biography}
%
%%% ----------------------------- Acknowledgements --------------------------- %%
\begin{acknowledgements}
I would like to thank Dr. Kowalsky for his help. The Staff and Students at The Constructed Facilities Lab (CFL) at NC State. Analytical Instrumentation Facility (AIF) at NC State. The Center for Additive Manufacturing and Logistics (CAMAL) at NC State. The Alaska Department of Transportation and Public Facilities (AKDOT P\&F).
\end{acknowledgements}


\thesistableofcontents

\thesislistoftables

\thesislistoffigures
