% ------------------------------ Abstract ---------------------------------- %
\begin{abstract}

Structures located in seismic regions are often subjected to multiple earthquakes. Multiple earthquakes can accumulate damage resulting in a deterioration of the seismic performance of a structure. In addition, aging of the structure can lead to corrosion that further propagates the deterioration. Past research has shown that the Park and Ang damage index (DI)(a demand parameter that quantifies damage) increases as damage aging conditions in the structures worsen or multiple seismic events are included in the analysis. The accumulation of damage increases the probability of structures collapsing. This research studied the effects of multiple earthquakes and damage accumulation in RC structures by developing strain-based fragility functions for different aging conditions. It was also vital to develop limit states representing corroded reinforcing steel to achieve this goal. Therefore, an experimental program was developed to perform accelerated corrosion in passivated reinforcing steel rebars. These corroded rebars were subjected to tension and buckled bar tension tests. The results from the experimental program showed the degradation in the performance of corroded reinforcing steel bars. The results also showed that the drop in the performance after mass loss corrosion level of 10\% was very significant. In addition, A framework incorporating corrosion models developed from the experimental program into a nonlinear time history analysis (NLTHA) was developed. In this analytical program, a series of SDOF cantilever columns was subjected to a sweep of earthquakes. The results show an increase in the probability of reaching a limit state when the corrosion level increases. The results from the analytical program corroborated that a limiting corrosion level of 10\% is ideal since the probability of reaching a strain limit state increased. Finally, this research proposed a new design and assessment methodology which will enable 1) to design structures that remain at a desired level of performance and 2) more accurately assess structures that have sustained a given level of corrosion. These methodologies are proposed for up to a maximum corrosion level of 10\%.


\end{abstract}


%% ---------------------------- Copyright page ------------------------------ %%
%% Comment the next line if you don't want the copyright page included.
\makecopyrightpage

%% -------------------------------- Title page ------------------------------ %%
\maketitlepage

%%% -------------------------------- Dedication ------------------------------ %%
\begin{dedication}
\centering
To God, ``How much better it is to get wisdom than gold! And to get understanding is to be chosen above silver'' (Prov 16:16). 
 \newline To my parents Victor M. Calderon and Roxana P. de Calderon, for all their love and support throughout the years and across the miles. 
 \newline To my wife Aubrey, ``If it takes doing a Ph.D. for us to meet it will have been worth it.'' Thank you for your love, support, patience, and for filling my days with joy. I love and cherish every moment, every laugh and everything we do. Te Amo Aubrey.
 \newline To all my friends and family, thank you for your support in this adventure.
\end{dedication}
%
%%% -------------------------------- Biography ------------------------------- %%
\begin{biography}
Victor A. Calderon was born in San Salvador, El Salvador. He obtained his Bachelor's in Civil Engineering at the Universidad Centroamericana José Simeón Cañas. Subsequently, he worked for the structural engineering consulting firm JEP Ingenieros y Arquitectos. He gained experience designing commercial buildings, schools, and residential buildings. While still part of JEP, he graduated with a Master's in Civil Engineering at NC State in 2016. That same year, he became part of a team that designed the facilities for a large dam project in El Salvador. In the spring of 2018, Victor decided to return to NC State for his Ph.D. program. In 2021, he became a registered Professional Engineer in North Carolina. He plans to continue his career as a structural engineer in the industry and serve his community.
\end{biography}
%
%%% ----------------------------- Acknowledgements --------------------------- %%
\begin{acknowledgements}
I would like to thank Dr. Kowalsky for his help. The staff and sudents at the Constructed Facilities Lab (CFL), the Analytical Instrumentation Facility (AIF), the Sensing and Materials Research Team (SMART) Lab, and the Center for Additive Manufacturing and Logistics (CAMAL) at NC State. Special thanks to the Alaska Department of Transportation and Public Facilities (AKDOT P\&F).
\end{acknowledgements}


\thesistableofcontents

\thesislistoftables

\thesislistoffigures
