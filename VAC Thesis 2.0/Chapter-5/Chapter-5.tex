\chapter{Analytical Program}
\label{chap-five}
This research developed an analysis procedure that incorporated the effect of cumulative damage in RC structures. While previous studies have obtained results using the Ang and Park Damage Index and obtained fragility curves using the assumptions of that model, this research evaluated the performance of structures using strain limit states since they provide a better estimate of the level of damage sustained by structures during an earthquake. Therefore, an analytical model of an SDOF cantilever RC column was subjected to a series of nonlinear time history analyses (NLTHA) that considered aging conditions and sequences of mainshock and aftershock. The analytical model incorporated the effective mechanical properties found in the experimental program.

The results from the analytical program showed that recorded ground motion sequences did not increase the strain demands. The maximum strain demands were achieved in all cases after a significant ground motion occurred. On the other hand, corrosion levels increased the demands and decreased the capacity of structures. Furthermore, the results show that as corrosion increases, a structure is more likely to achieve a given limit state for a prescribed earthquake intensity level. In addition, from the NLTHA results, it was determined that for a level of corrosion is CL=10\%, the increase in the likelihood of reaching a limit state increased, which was congruent with the experimental program results. Therefore, a proposed maximum level of corrosion of 10\% is recommended.

\section{Modeling of Corrosion for Structural Analysis}

The main objective of the analytical program is to determine the effects of corrosion on the demands of reinforced concrete columns. The results obtained in the experimental program have shown that corrosion affects the geometry of the surface of reinforcing steel bars by generating imperfections. The geometrical imperfections induced by the corrosion caused a reduction in the effective mechanical properties of the reinforcing steel material. The effective mechanical properties are a mathematical convenience to evaluate corroded RC structures. In general, as corrosion increases, the effective material properties of the steel decrease, as well as the average bar diameter.

In order to incorporate the effect of corrosion in the structural analysis, it is necessary to reduce the bar diameter and the effective mechanical properties of the reinforcing steel bar. For this study, uniform corrosion is the only type of corrosion considered. Therefore, assuming uniform corrosion, the diameter is reduced per the following expressions:

\begin{equation}
    d_{b,CL} = d_{b,o} \sqrt{1 - CL*0.01}
    \label{eq:d_eff}
\end{equation}

Where CL corresponds to the corrosion level, and $d_{b,CL}, d_{b,o}$ are the reduced diameter and the initial diameter correspondingly. 

Similarly, the mechanical properties of the reinforcing steel are modified with the expressions for effective yield and ultimate strengths found in the experimental program. The expressions proposed in Chapter \ref{chap-four} are replicated here for convenience:

\begin{equation}
    f_{y,CL} = f_{y,o}(1-0.0075CL)
    \label{eq.Calderon_Fy_vs_CL_5}
\end{equation}

\begin{equation}
    f_{u,CL} = f_{u,o}(1-0.0075CL)
    \label{eq.Calderon_Fu_vs_CL_5}
\end{equation}

In addition, the ultimate limit state uses the maximum bending strain to calculate the previous tension strain for buckled bar fracture. Since the analytical program evaluates different corrosion levels, \eref{eq.Calderon_eb_vs_CL} which defines the degradation of the maximum bending strain with corrosion level is used. \eref{eq.Calderon_eb_vs_CL} is replicated here for convenience as \eref{eq.Calderon_eb_vs_CL_02}

\begin{equation}
    \varepsilon_{b}(CL) = \varepsilon_{o}-0.0045CL
    \label{eq.Calderon_eb_vs_CL_02}
\end{equation}

\section{Multiple Seismic Events}

The evaluation of multiple seismic events is a topic that has been scarcely studied. However, their effects have been felt in numerous earthquake sequences, such as the Christchurch 2010, Umbria-Marche Earthquake 1997, and the Puerto Rico Earthquakes 2020. The hypothesis was that after a mainshock event, the accumulation of damage would restart in a minor seismic event and achieve a higher limit state than the limit state achieved in the mainshock.

This study determined that not all damage in structures is dependent on multiple events. The condition of the structure when the event occurs must also be quantified. Therefore, in this study, the sequences of mainshock and aftershock (MS-AS) were evaluated at different levels of corrosion.

\subsection{Earthquake Selection}

The ground motions were taken from the NGA2 West Database of earthquake records provided by the Pacific Earthquake and Engineering Research Institute (PEER) \cite{Ancheta2014}. This database consists of 2578 different earthquake events that characterize shallow crustal earthquakes. To study the effect of mainshock and aftershock sequences, only records that had aftershocks recorded were used. In addition, the parameters shown below were also used to filter the database to ensure that the records would produce inelastic displacement in the structures being analyzed.

\begin{itemize}
	\item Moment magnitude $M_w \geqslant 5$
	\item $PGA>0.04$
	\item $PGV>1$ cm/s
	\item $Vs_{30}>100m/s$ \& $Vs_{30}<1000$ m/s
	\item Lowest usable frequency is less than 1Hz
	\item $R_{rup}<60km$
\end{itemize}

Figure \fref{fig:GM_Selection} summarizes the results from filtering the data available in the PEER NGA West2 database which resulted in 456 aftershocks and 375 mainshocks. \fref{fig:GM_Selection} shows the earthquakes as moment magnitude {Mw} vs rupture distance ($R_{rup}$). In addition, the spectral displacement is an important intensity measure used in this study. Therefore, the displacement spectrum for the ground motions used in this research are shown in \fref{fig:DisplacementSpectrum_Selection}. Figures \fref{fig:GM_Selection} and \fref{fig:DisplacementSpectrum_Selection}, show that there is a good level of variability between the records and provide a wide range of ground motion properties. 

\begin{figure}[htbp]
	\centering
	\includegraphics[width=0.45\textwidth]{Chapter-5/figs/GM_Selection.pdf}
	\caption{Mainshock selection from PEER NGA West2 database}
	\label{fig:GM_Selection}
\end{figure}

\begin{figure}[htbp]
	\centering
	\includegraphics[width=0.45\textwidth]{Chapter-5/figs/SD_Spectrum_GM_Selection.pdf}
	\caption{Mainshock selection from PEER NGA West2 database}
	\label{fig:DisplacementSpectrum_Selection}
\end{figure}

\subsection{Modeling of Mainshock-Aftershock Series for Corrosion}

In order to analyze the structure of a sequence of mainshock and aftershock, it was necessary to clip together two ground motions. The process consisted of 1) selecting mainshocks and aftershocks from the same recording station and assigned to the same earthquake sequence, 2) the records were combined into a single file to run the sequence in order. Thus, a 4-second gap of zero acceleration was added between the mainshock and the aftershock. The 4-second gap between the mainshock and aftershock ground motions represents the situation in which the structure comes to rest after the first event. Although the time to reach the at-rest position can vary depending on the structural and ground motion properties, the recommendation from Raghunandan et al. \cite{Raghunandan2015} found that 4-seconds gave enough time for a structure to reach the at-rest position for most single degree of freedom systems while keeping computational costs low. A more significant time gap is possible, but it increases the computational cost significantly. An example of the resulting mainshock and aftershock sequence is shown in \fref{fig:MS-AS_sequence_sample}.

\begin{figure}[htbp]
	\centering
	\includegraphics[width=0.55\textwidth]{Chapter-5/figs/MS_AS_Figure.pdf}
	\caption{Mainshock selection from PEER NGA West2 database}
	\label{fig:MS-AS_sequence_sample}
\end{figure}

\section{Analytical Model}
\subsection{SDOF System: Cantilever Column}
This study focused on the behavior of a single degree of freedom (SDOF) system representing a cantilever reinforced concrete column. The column is modeled as shown in \fref{fig:Structural_Model} This structure is modeled in OpenSeesPy \cite{McKenna2010}\cite{Zhu2018} using the $forceBeamColumn$ element \cite{Scott}. The forceBeamColumn element is used with two-point Gauss-Radau integration applied in the hinge regions, and two-point Gauss integration applied on the element interior for a total of six integration points \cite{Scott}. The force-based formulation requires only a single element to represent the full nonlinear deformation of the member accurately, and the integration scheme selected prevents the loss of objectivity during the softening response while also providing integration points at the member ends \cite{Calabrese2010},\cite{Scott}. The element requires the length of plasticity to be defined at each end of the member, for which the tension-based rectangular plastic hinge length is calculated using the following expressions \cite{Goodnight2013}:
\begin{equation}
    L_{pc}=k*L_{eff} + 0.4D
    \label{eq:LP_Comp}
\end{equation}
\begin{equation}
	k=0.2*(Fu/Fy - 1) \leqslant 0.08
	\label{eq:K_Lp}
\end{equation}
\begin{equation}
    L_{pt}=L_{pc}+\gamma*D
    \label{eq:LP_Tension}
\end{equation}

For single bending the parameter $\gamma$ is:
\begin{equation}
    \gamma=0.33
    \label{eq:Gamma_LPt}
\end{equation}

The two-point Gauss-Radau integration was applied such that each end node integration is weighted equal to the specified plastic hinge length, as illustrated in \fref{fig:Fiber_PlasticHinge}. In this figure, $D$ is the diameter of the column, and $c$ is the concrete cover. Therefore, strains recorded at the end sections represented accurate values even when deformation localizes to the ends from strain-softening behavior. For the case of the cantilever column, only one plastic hinge length is defined, and the opposite end is given an arbitrary unit length. 

\begin{figure}[htbp]
	\centering
	\includegraphics[width=0.75\textwidth]{Chapter-5/figs/StructuralModel_01}
	\caption{Structural Model: a) SDOF Column b) Fiber Model Representation}
	\label{fig:Structural_Model}
\end{figure}

\begin{figure}[htbp]
	\centering
	\includegraphics[width=0.9\textwidth]{Chapter-5/figs/fbc_PlasticHinge}
	\caption{End point plastic hinge method \cite{Scott}}
	\label{fig:Fiber_PlasticHinge}
\end{figure}

The cross section of the column is shown in \fref{fig:ColumnSection}. The column cross section is discretized with concrete and steel material fibers. Concrete fibers are modeled using the $Concrete01$ material, modified for confined material strength based on the Mander confined concrete model \cite{Mander1988}. The $Steel02$ material, based on the Giuffre-Menegotto-Pinto model \cite{Filippou1983} is used for the longitudinal reinforcement with recommended parameters ($b = 0.01, R0 = 20, cR1 = 0.925, cR2 = 0.15$). 

\begin{figure}[htbp]
	\centering
	\includegraphics[width=0.7\textwidth]{Chapter-5/figs/StructuralModel_Section}
	\caption{Section of the RC Column}
	\label{fig:ColumnSection}
\end{figure}
\subsection{Strain Penetration}

The strain penetration considers the additional deformation due to anchorage of the reinforcement into the foundation, since tension strain in the reinforcement will drop to zero at a depth equal to the true development length of the rebar \cite{Priestley2007}. Experimental studies have generally reported that this end rotation contributes up to 35\% to the lateral deformation of flexural members\cite{Zhao2007}. Therefore, it is important to incorporate it into the analytical model. A way to capture this effect is by using a zero-length section element implemented in the nonlinear fiber-based analysis of concrete structures, which is available in the material library of OpenSeesPy as $Bond SP1$ \cite{Zhao2007}. This is the material model used for the steel fibers of the zero-length section element.

The required parameters for this model are:
\begin{itemize}
	\item $F_{y}$ Yield strength of the reinforcement steel
	\item $S_{y}$ Rebar slip at member interface under yield stress (see \eref{eq.Rebar_Slip})
	\item $F_{u}$ Ultimate strength of the reinforcement steel
	\item $S_{u}$ Rebar slip at the loaded end at the bar fracture strength a value of $35 S_{y}$ is recommended \cite{Zhao2007}
	\item $b$ Initial hardening ratio in the monotonic slip vs. bar stress response $b=0.45$ is recommended \cite{Zhao2007}
	\item $R$ Pinching factor for the cyclic slip vs. bar response $R=1.01$ is recommended \cite{Zhao2007}
	\item $d_b$ Rebar diameter
	\item $f'c$ Concrete compressive strength of the adjoining connection member
	\item $\alpha$ Parameter used in the local bond-slip relation and can be taken as $\alpha=0.4$ in accordance with CEB-FIP Model Code 90 \cite{CEB1993}
\end{itemize}.
\newline
Bar slip is calculated as:
\begin{equation}
	S_{y}(in)=0.1\left(\frac{d_{b}F_{y}}{4000\sqrt{f'_{c}}}\left(2\alpha+1\right)\right)^{\frac{1}{\alpha}}+0.013 (in)
	\label{eq.Rebar_Slip}
\end{equation}
\subsection{Design Limit States}
Design limit states are defined based on strains in the material since they can more accurately represent the different performance levels. Structure limit states are tension strains in the rebars or compression strains in the concrete core. It has been shown that tension strain limits are reliable indicators of structural damage \cite{Goodnight2016}. The values used in this study for the performance-based analysis of reinforced concrete bridge columns are shown in Table \ref{tab:DesignLimitStates}. The serviceability limit states correspond to the compression strain at which concrete cover begins to crush and the peak tension strain, resulting in residual crack widths of approximately 1 mm. These limits are generally accepted as nominal limit states for RC members. The compression limit state for damage control is defined by the expression shown in \eref{eq:ec_DamageControl}, and it refers to the compression strain in the confined concrete, at which fracture of the transverse reinforcement confining the core occurs \cite{Priestley2007}. This equation is obtained using the strain-energy balance between that absorbed by the confined core concrete and the capacity of the confining steel. The damage control limit state is defined by the strain at the onset of buckling, which can be expressed according to \eref{eq:es_DamageControl}. This model demonstrated accurate predictions of the onset of bar buckling on physical tests in SDOF concrete column \cite{Goodnight2016}. Finally, the ultimate limit state was taken from Barcley et al. \cite{Barcley2019}. The ultimate limit state is the previous tension strain before bar fracture due to bar buckling. The limit state can be expressed as shown in \eref{eq:es_ultimate}. For the case of corrosion, the effective material properties of the reinforcing steel bars were used, and in the case of the ultimate limit state, the equation that relates corrosion and the maximum bending strain was used. It must be noted that \eref{eq:ec_DamageControl}, \eref{eq:es_DamageControl} have been recommended for the AASHTO Guidelines for Performance Based Seismic Bridge Design \cite{eq:es_DamageControl}.

\begin{equation}
    \varepsilon_{c,spiral yield}=0.009-0.3\frac{A_{st}}{A_{g}} +3.9\frac{f_{yhe}}{E_{s}}
    \label{eq:ec_DamageControl}
\end{equation}

\begin{equation}
    \varepsilon_{s,BB}=0.03+700\rho_{s}  \frac{f_{yhe}}{E_{s}} -0.1\frac{P}{f'_{c}A_{g}}
    \label{eq:es_DamageControl}
\end{equation}
\begin{equation}
    \varepsilon_{t}=\frac{ln(\frac{\varepsilon_{b}}{0.001})}{\frac{300P}{f'c A_{g}}+\frac{0.7}{\rho_{t}}}
    \label{eq:es_ultimate}
\end{equation}

\begin{table}[htpb]
	\caption{Design limit states}
	\label{tab:DesignLimitStates}
        \begin{center}
        \begin{tabular}{lll}
        Limit State         & Concrete Limit State (in/in) & Reinforcing Steel Limit State (in/in) \\ \hline
        Serviciability      & 0.04                         & 0.015                                 \\ 
        Collapse Prevention & \eref{eq:ec_DamageControl}   & \eref{eq:es_DamageControl}             \\ 
        Fracture            & N/A                          & \eref{eq:es_ultimate}                   \\ 
        \end{tabular}
        \end{center}
\end{table}

\section{Comparison with Existing Physical Tests}
The model used in this research was calibrated for the case of pristine conditions and the case of corroded columns. First, the calibration to a pristine condition column shows how reliable the results from the structural model are. Then the pristine condition model is modified with the corrosion model, as explained in section 5.1. Finally, the analytical model is compared to the results from the physical test on corroded RC columns. The analytical model confirmed that the results obtained from the analytical program are reliable.
\subsection{Pristine Condition Column}
Goodnight et al performed a total of 30 circular RC columns quasi-static tests to evaluate strain limit states \cite{Goodnight2016}. From this set of tests, a sample of one was taken to calibrate the analytical model. The test performed by Goodnight et al on an SDOF cantilever column shows similar geometry to that presented in \fref{fig:Fiber_PlasticHinge}. The parameters used in this large scale test were: diameter $D = 24.0 inch$, height of the column $L = 8.0 ft$, yield strength of steel $f_{y} = 83 ksi$, ultimate strength of steel $f_{u} = 109 * ksi$, longitudinal steel volumetric ratio $\rho_{s} = 1.5\% $, transverse steel volumetric ratio $\rho_{v} = 1.0\% $, and strength of concrete at 8 days $f'_{c} = 6 ksi$.

The analytical model used these parameters to compare the results from the model to the experimental results. The results from the analysis show good agreement with the experimental results as evidenced in \fref{fig:ModelCalibration}. Thus, the results obtained from the model predict the overall system behavior and can be used to analyze other configurations of the structural model.

\begin{figure}[htbp]
	\centering
	\includegraphics[width=0.60\textwidth]{Chapter-5/figs/Calibration_Test_26_Goodnight_et_al.pdf}
	\caption{Force-Displacement results from experimental results \cite{Goodnight2013} and analytical model}
	\label{fig:ModelCalibration}
\end{figure}

In addition, to verify the applicability of the strain limit states defined in the previous section, the limit states that correspond to bar buckling ($\varepsilon_{s,bb}$ ) and fracture of buckled bar ($\varepsilon_{t}$ )  were evaluated for this test. From \fref{fig:ModelCalibration_Pristine_Hysteresis} if the strain hysteresis is intercepted by each of the strain limits states, it can be observed that for the bar buckling limit state, the displacement corresponds to a 5.5 inch, it is at this displacement in \fref{fig:ModelCalibration} that the first drop in the strength of the system is observed. Similarly, for buckled bar fracture, the displacement corresponds to 5.8, which is where there is a substantial drop in the strength of the column due to fracture of the buckled bar in \fref{fig:ModelCalibration}. Thus, the analysis can capture the limit states in the structure. 

\begin{figure}[htbp]
	\centering
	\includegraphics[width=0.5\textwidth]{VAC Thesis 2.0/Chapter-5/figs/Calibration_Test_26_Goodnight_et_al_strain.pdf}
	\caption{Strain hysteresis from experimental RC column in pristine condition}
	\label{fig:ModelCalibration_Pristine_Hysteresis}
\end{figure}

\subsection{Accelerated Corrosion Column}
For the case of RC columns with accelerated corrosion, Ma et al performed a series of quasi-static tests on RC columns with different corrosion levels and axial load ratios \cite{Ma2012}. From their study, the test with a corrosion level $CL=9.5\%$ was taken for calibration since the other tests presented in their study had excessively high axial load ratios which are not common in RC bridges. The results from the Ma et al test \cite{Ma2012} were used to compare against the analytical model. The column had the following parameters: diameter $D = 260.0 mm$, height of the column $L = 820.0 mm$, yield strength of steel $f_{y} = 375.0$ MPa, ultimate strength of steel $f_{u} = 572.3$ MPa, longitudinal steel area ratio $\rho_{l} = 2.73\% $, transverse steel volumetric ratio $\rho_{v} = 1.0\% $, strength of concrete at 8 days $f'_{c} = 39.8$ MPa, and axial load ratio of $ALR=15\%$. Equation \ref{eq.eleven} is used to modify the material properties of the reinforcing steel and considers the effects of corrosion. 

\begin{figure}[htbp]
	\centering
	\includegraphics[width=0.50\textwidth]{Chapter-5/figs/Model_vs_MaEtAl_220218.pdf}
	\caption{Force-Displacement results from experimental RC column with corrosion in longitudinal bar (CL=9.5\%) \cite{Ma2012} and analytical model results (shown in light blue)}
	\label{fig:ModelCalibration_Corrosion}
\end{figure}

Figure \ref{fig:ModelCalibration_Corrosion} shows that the results obtained from the analytical model capture the response of the structure with reasonable accuracy up to a displacement of $\pm 33 mm$. The observed discrepancies are due to the inability of fiber models to represent buckling in the longitudinal reinforcing steel and the lack of strain data and material quantification from the authors' study \cite{Ma2012}. In addition, Ma et al. \cite{Ma2012} did not report if bar buckling and bar fracture occurred during the test. However, the hysteresis curve shown in their study suggests that some damage limit state was reached. 

In order to prove that \eref{eq:es_DamageControl} can predict buckling relatively close for corroded RC Columns, the strain at the extreme longitudinal bar from the analytical model is compared with the value obtained using the equation limit state\eref{eq:es_DamageControl}. Using \eref{eq:es_DamageControl} the bar buckling limit state is calculated as ($\varepsilon_{s,BB}=0.016$). Intersecting  $\varepsilon_{s,BB}=0.019$  with the strain hysteresis from the analytical model results shown in \fref{fig:ModelCalibration_Corrosion_Hysteresis}, the displacement that corresponds to this limit state was $36 mm$ which is higher than the -33mm obtained from the test results. The value obtained using \eref{eq:es_DamageControl} is within 10\% of the value obtained from the test results. Therefore, the analytical model results are appropriate for the scope of this research. The reasons for the differences might be related to the effect corrosion has on RC columns which were out of the scope of Goodnight et al. in the development of \eref{eq:es_DamageControl}.

Finally, for the buckled bar fracture limit state, equation \eref{eq.Calderon_eb_vs_CL_02} and the effective mechanical properties expression \eref{eq.Calderon_Fy_vs_CL_5} and \eref{eq.Calderon_Fu_vs_CL_5} were obtained for the test data provided by \cite{Ma2012}. The limit state calculated using \eref{eq:es_ultimate} was $\varepsilon_{s}=0.023$. The strain predicts the fracture of the bars reported by Ma et al. \cite{Ma2012} at a displacement of $-50 mm$, as shown in \fref{fig:ModelCalibration_Corrosion_Hysteresis}. While the result obtained from the analytical model developed in this research seems to predict with relative accuracy the performance of corroded RC columns reported by Ma et al. \cite{Ma2012}, ongoing research at NC State will further improve the estimation of the limit states for corroded RC columns.

\begin{figure}[htbp]
	\centering
	\includegraphics[width=0.5\textwidth]{VAC Thesis 2.0/Chapter-5/figs/Calibration_Ma_et_al_strain.pdf}
	\caption{Strain hysteresis from experimental RC column with corrosion in longitudinal bar (CL=9.5\%) results from analytical model}
	\label{fig:ModelCalibration_Corrosion_Hysteresis}
\end{figure}

\section{Intensity Measure: $Sd(T_{eff},\xi)$}

When relating ground motions to structural response parameters, selecting appropriate quantities that accurately capture their relationship is crucial. Krish \cite{Krish2018} showed in a recent study that there is a good correlation between strain obtained from fiber modeling and first mode effective spectral displacement ($IM=Sd(T_1)$). On the other hand, peak ground acceleration $(PGA)$ did not correlate well. These conclusions are congruent with the results found by Mackie et al. \cite{Mackie2003}. 

In this study, the intensity measure was improved further by correlating the strains to the effective period of the structure $(T_{eff}$ and the equivalent damping $(\xi)$. These parameters are of substantial use in the Direct Displacement Based Design Methodology. The calculation of the effective period of the structure and the equivalent damping are explained below.

\subsection{Effective Period Calculation}

The process to  calculate the effective period consists of obtaining first the effective stiffness of the structure. From the Non-Linear Time History Analysis (NLTHA) of a structure, the maximum displacement and force at the maximum displacement are obtained, as shown in \fref{fig:k_eff_calculation}. With these values, the effective stiffness can be calculated as:

\begin{equation}
     K_{eff}=\frac{F(d_{max})}{d_{max}}
    \label{eq:Keff_calcualtion}
\end{equation}

\begin{figure}[htbp]
	\centering
	\includegraphics[width=0.60\textwidth]{VAC Thesis 2.0/Chapter-5/figs/Force_Diplacement_Keff_Calc.pdf}
	\caption{Calculation of effective stiffness $(k_{eff})$}
	\label{fig:k_eff_calculation}
\end{figure}

After obtaining the effective stiffness, the effective period can be calculated with the relationship for the period of a structure. The effective period is calculated as:

\begin{equation}
     T_{eff}=2\pi \sqrt{\frac{M}{K_{eff}}}
    \label{eq:Teff_calcualtion}
\end{equation}

In the analytical program the mass is calculated on the basis of the axial load applied to the structures. 

\begin{equation}
    M=\frac{P}{g}
    \label{eq:M_calcualtion}
\end{equation}

\subsection{Calculate $Sd(T_{eff},\xi)$}

After obtaining the effective period of the structure for each ground motion, it is possible to obtain the spectral displacement at 5\% damping for each ground motion for a given effective period. Finally, the equivalent damping that the system reached for a given ground motion must be found to obtain the spectral displacement at the effective period and equivalent damping. Using the expressions for equivalent damping from Priestley et al. for circular columns in bridges, the equivalent damping can be expressed as:

\begin{equation}
    \xi_{eq}=0.05+0.444\frac{\mu-1}{\mu\pi}
    \label{eq:EqDamping_calcualtion}
\end{equation}

The damping factor that corresponds to the equivalent damping is calculated as:

\begin{equation}
    DF=\sqrt{\frac{0.07}{0.02+\xi_{eq}}}
    \label{eq:DF_calcualtion}
\end{equation}

Equations \ref{eq:EqDamping_calcualtion} and \ref{eq:DF_calcualtion} are part of the DDBD procedure outlined by Priestley et al \cite{Priestley2007}. Finally, the spectral displacement at effective period and equivalent damping can be expressed as:

\begin{equation}
    Sd(T_{eff},\xi)=DF \times Sd(T_{eff},5\%)
    \label{eq:Sd_teff_xi_calcualtion}
\end{equation}

The methodology to obtain $Sd(T_{eff},\xi)$  can be seen graphically in \fref{fig:SpectralDisplacementCalculation}.

\begin{figure}[htbp]
	\centering
	\includegraphics[width=0.6\textwidth]{VAC Thesis 2.0/Chapter-5/figs/SpectralDisplacement_SD(Teff,xi)_Calc.pdf}
	\caption{Calculation of spectral displacement at effective period at 5\% damping $Sd(T_{eff},5\%)$ and at equivalent damping $Sd(T_{eff},\xi)$}
	\label{fig:SpectralDisplacementCalculation}
\end{figure}

\section{Analysis of Results Using MSA}

The Multiple Stripe Analysis (MSA) procedure was used to fit fragility curves to the raw data, as described by Baker \cite{Baker2015}. The MSA procedure consists of: First, at each intensity level, $x_j$, the probability, $P$, of observing $z_j$ collapses in $n_j$ observations is given by the binomial distribution, as shown in \eref{eq:P_MSA}, where $p_j$ is the probability of a single ground motion with $IM = x_j$ to cause the collapse of the structure, this is expressed as:

\begin{equation}
   P=\binom{n}{z_{j}}p_{j}^{z_{j}}(1-p_{j})^{n_{j}-z_{j}} 
   \label{eq:P_MSA}
\end{equation}

Then, A collapse in the MSA methodology indicates that the specified limit state has been exceeded. Afterward, to consider the overall likelihood of observing multiple instances of this distribution across multiple IM levels, the product of this distribution is taken for each IM level considered and can be expressed mathematically as:

\begin{equation}
   Likelihood=\prod_{j=1}^{m} \binom{n}{z_{j}}p_{j}^{z_{j}}(1-p_{j})^{n_{j}-z_{j}} 
   \label{eq:likelihood_MSA}
\end{equation}

Finally, a lognormal cumulative distribution function (CDF) is used to define the fragility function and is substituted for $p_j$. The fragility function parameters are defined as the median of the fragility function, $\theta$, and the standard deviation of $ln(IM)$, $\beta$ (also referred to as the dispersion of IM). Hence, the solution is simplified by taking the natural logarithm of each side, and the product formulation in\eref{eq:likelihood_MSA} is reduced to a summation that is computationally more efficient. The resulting function is expressed as:

\begin{equation}
  \{\hat{\theta},\hat{\beta}\}=\arg \max_{\theta,\beta} \sum_{j=1}^{m} \bigg\{\ln\binom{n}{z_{j}} +z_{j}\ln\Theta\left(\frac{\ln(x_{j}/\theta)}{\beta}\right) + (n_{j}-z_{j})\ln\left[1-\Theta\left(\frac{ln(x_{j}/\theta)}{\beta}\right)\right]\bigg\} 
  \label{eq:hat_theta_beta_MSA}
\end{equation}

In \eref{eq:hat_theta_beta_MSA}, $n_j$, $z_j$, and $x_j$ are all defined via the analysis results, and $\theta$ and $\beta$ are the only unknowns. These two unknown parameters are then optimized using numerical solution techniques to find the values that produce the highest probability of observing the analysis results. The Excel spreadsheets and MatLab codes, developed by Baker \cite{Baker2015}, were used to produce the analysis of the results.

The MSA procedure was selected since it is well suited for unscaled ground motions, which are the types of ground motions used in this study. An example of curves fit the analysis data for the limit state of damage control for steel and corrosion level of 10\% is shown in \fref{fig:msa_sample_01}

\begin{figure}[htp]
	\centering
	\includegraphics[width=0.60\textwidth]{VAC Thesis 2.0/Chapter-5/figs/MSA_Calc.pdf}
	\caption{MSA analysis}
	\label{fig:msa_sample_01}
\end{figure}
\section{Analytical framework}

The analytical framework was  established to obtain the change in the structure performance   due to aging conditions and to evaluate the effect of seismic events  on the achievement of strain limit states for  single degree of freedom columns. This framework consisted of a program that performed  and analyzed a series of nonlinear time history analyses (NLTHA). From these analyses, it was then possible to the effects of condition on the achievement of different performance limit states for a given seismic intensity. The proposed analytical framework process consisted of:

\begin{enumerate}
	\item Read the cross sectional properties of the SDOF column from the database of columns(12 different cross sectional properties). Read the Aspect Ratio ($L/D=[4,6,8]$), read the axial load ratio ($ALR=[5\%,10\%,15\%,20\%$]), read the level of corrosion ($CL=[0\%,5\%,10\%,15\%,20\%]$).This resulted in 168 different columns.
	\item Evaluate, the effective mechanical properties of the reinforcing steel for each of the corrosion levels evaluated using.
	\item Perform the Non Linear Time History Analysis (NLTHA) of discrete ground motions and sequence of ground motions
	\item Evaluate the results from the NLTHA to post-process the data and store it in a database for easy access in the MSA analysis.
\end{enumerate}

The analysis matrix for the corrosion aging phenomenon that was analyzed in this study is shown in Table \ref{tab:AnalysisMatrix}. A total of 54 Mainshocks and 405 Mainshock-Aftershock sequences were used in the analysis. The area or extent covered in the analysis corresponds to the range of variables that are common for RC columns in bridges.This analysis matrix resulted in a total of 36,000 analyses for the discrete ground motion analysis and 96,000 analyses for the sequences of ground motion case. To perform this large volume of analyses the Henry2 High Performance Computer (HPC) at NC State was used to run the analyses in parallel. Table \ref{tab:column_geometry} shows the detailing and column diameters used in the analytical program.

\begin{table}[htb]
	\caption{Analysis matrix}
	\label{tab:AnalysisMatrix}
	\centering
\begin{tabular}{{lcc}}
Parameter                          & Parameter        & Range                  \\	\hline
Diameter of column                     & D                & 28-90 in               \\	
Column aspect ratio        & L/D              & 4-8                    \\	
Longitudinal ratio                     & $\rho_s$         & 0.01-0.04              \\	
Axial load ratio                       & ALR              & 5\%-20\%               \\	
Corrosion level                         & CL               & 0\%-20\%               \\	
\end{tabular}
\end{table}


\begin{table}[htpb]
\caption{Detailing of columns for NLTHA}
\label{tab:column_geometry}
\begin{tabularx}{1.0\textwidth} { 
   >{\centering\arraybackslash}X 
   >{\centering\arraybackslash}X 
  >{\centering\arraybackslash}X >{\centering\arraybackslash}X >{\centering\arraybackslash}X >{\centering\arraybackslash}X >{\centering\arraybackslash}X}
Diameter, $D (in)$ & Bar diameter, $d_{b} (in)$ & Number of bars, $n_{b}$ & Trans. diameter, $d_{t} (in) $ & Spacing, $s (in)$ & $\rho_{l}$ & $\rho_{v}$ \\ \hline
48               & 0.875               & 30                             & 0.625                & 2.5                   & 0.01   & 0.01   \\
48               & 1.000                   & 45                             & 0.625                & 2.5                   & 0.02   & 0.01   \\
48               & 1.125               & 54                             & 0.625                & 2.5                   & 0.03   & 0.01   \\
48               & 1.250                & 56                             & 0.625                & 2.5                   & 0.04   & 0.01   \\
72               & 1.000                   & 52                             & 0.75                 & 2.5                   & 0.01   & 0.01   \\
72               & 1.125               & 82                             & 0.75                 & 2.5                   & 0.02   & 0.01   \\
72               & 1.250                & 96                             & 0.75                 & 2.5                   & 0.03   & 0.01   \\
72               & 1.750                & 72                             & 0.75                 & 2.5                   & 0.04   & 0.01   \\
90               & 1.000                   & 80                             & 0.75                 & 2                     & 0.01   & 0.01   \\
90               & 1.125               & 128                            & 0.75                 & 2                     & 0.02   & 0.01   \\
90               & 1.250                & 150                            & 0.75                 & 2                     & 0.03   & 0.01   \\
90               & 1.750                & 114                            & 0.75                 & 2                     & 0.04   & 0.01  
\end{tabularx}
\end{table}


\subsection{Analysis Algorithm}

In order to run the analysis efficiently, a program was developed to perform three main routines:
1) Main program: setting conditions, the geometry of the model, effective material properties, 2) Run the Non-Linear Time History Analysis, 3)Post-processing of data

The main program is shown in \fref{fig:main_flowchart}. This program has three main inputs the ground motion records, the geometry of the column, and the aging conditions. The aging conditions relate include the corrosion level, the initial material properties, the axial load ratio, and the aspect ratio. After the inputs are selected, a nested loop goes through the different parameters.  The basic flow consists of submitting the data to the NLTHA subroutine and the post-processor subroutine. Once the program goes through the subroutines, the data output from OpenSees is deleted to use the HPC resources efficiently. This process is repeated until all the variables have been evaluated and the program finishes.

\begin{figure}[htp]
	\centering
	\includegraphics[width=0.975\textwidth]{VAC Thesis 2.0/Chapter-5/figs/Main_FlowChart_01.pdf}
	\caption{Main Flow-Chart}
	\label{fig:main_flowchart}
\end{figure}

The NLTHA subroutine consisted of a sequential process as shown in \fref{fig:nltha_flowchart}. First, the geometry of the model is established as per the model shown in \fref{fig:Structural_Model}. Then the material properties and the cross-sectional fibers are defined per the details shown in \ref{tab:column_geometry}. Next, the recorders that store the analysis results throughout the NTLHA are set. Then the axial load is run and kept constant throughout the analysis. Finally, the NLTHA runs until it finishes.

\begin{figure}[htp]
	\centering
	\includegraphics[width=0.40\textwidth]{VAC Thesis 2.0/Chapter-5/figs/NLTHA_FlowCharts_01.pdf}
	\caption{NLTHA Flow-Chart}
	\label{fig:nltha_flowchart}
\end{figure}

The post-processor subroutine also consisted of a sequential process as shown in \fref{fig:postproc_flowchart_01} and \fref{fig:postproc_flowchart_02} . The main goal of this subroutine is to store the data related to the model, including the geometry and material properties used in the analysis. The post-processor calculates the limit states as explained in section 5.3.3. Then the spectral displacement at the effective period is calculated for the ground motion following the procedure explained in section 5.5. Finally, a collapse analysis for each limit state is performed in order to perform the multi-stripe analysis explained in the following section. All the relevant data is then stored in a database for further analysis.

\begin{figure}[htp]
	\centering
	\includegraphics[width=0.575\textwidth]{VAC Thesis 2.0/Chapter-5/figs/PostProcessor_FlowCharts_01.pdf}
	\caption{Post-processor Flow-Chart}
	\label{fig:postproc_flowchart_01}
\end{figure}

\begin{figure}[htp]
	\centering
	\includegraphics[width=0.75\textwidth]{VAC Thesis 2.0/Chapter-5/figs/PostProcessor_FlowCharts_02.pdf}
	\caption{Post-processor Flow-Chart continued}
	\label{fig:postproc_flowchart_02}
\end{figure}

\section{Results from Analytical Program}
This section presents the results obtained from a non-linear time history analysis (NLTHA) performed using OpenSeesPy \cite{Zhu2018}. The structures were subjected to a total of 54 earthquake mainshock records and where applicable 405 Mainshock-Aftershock sequences. The mainshock records displacement spectrum are shown in \fref{fig:DisplacementSpectrum_Selection}, and a list of the records is shown in Appendix C. The primary responses obtained from these analyses correspond to the maximum strain obtained for the different levels of corrosion. The structures were analyzed for the range of corrosion levels [0\%-20\%] in the longitudinal rebars.

\subsection{Structural Response at Different Corrosion Levels}
Figures \ref{fig:Force-Displacement_Results} and \fref{fig:Steel_Stress_Strain_Response} are presented as an example of the results obtained using NLTHA. The column shown in this example had the following details: 
\begin{itemize}
    \item Diameter, $D = 48in$
    \item Aspect ratio, $L/D=4$
    \item Longitudinal Bar Diameter, $d_{b}= 0.875 in$; Number of Longitudinal Bars, $n_{b} = 30$
    \item Transverse Bar Diameter, $d_{t} = 0.625 in$; Spacing of transverse steel, $s = 2.5 in$
    \item Axial Load Ratio, $ALR = 10\%$
    \item Corrosion Level, $CL = [0\%,5\%,10\%,15\%,20\%]$ 
    \item Pristine Condition yield strength of reinforcing steel, $F_{y}=60 ksi$
    \item Concrete strength at 28 days, $f'c = 5 ksi$
\end{itemize}

Here results are extracted from the structure's response to the Chi-Chi earthquake (RSN1505 in the PEER database). \fref{fig:Force-Displacement_Results} shows the global system response. It can be observed that as the corrosion level increases, the displacement demands also increase.
On the other hand, it is known that corrosion tends to degrade the capacity of an RC system, as will be shown in section 5.8.3. As corrosion increases, the likelihood of reaching a limit state also increases. Similarly, in \fref{fig:Steel_Stress_Strain_Response} and \fref{fig:Steel_Strain_Response} shows first that the strength of the corroded reinforcement decreases and therefore induces higher strain demands on the system, and the strain demands from the same ground motion increases as the corrosion level increases. These results further illustrate the effect corrosion could have on an RC structure.

\begin{figure}[htbp]
	\centering
	\includegraphics[width=0.7\textwidth]{Chapter-5/figs/Force_Diplacement_RSN1505.pdf}
	\caption{Force-Displacement results}
	\label{fig:Force-Displacement_Results}
\end{figure}

\begin{figure}[htbp]
	\centering
	\includegraphics[width=0.7\textwidth]{Chapter-5/figs/Stress_Strain_RSN1505.pdf}
	\caption{Stress strain response for extreme rebar location}
	\label{fig:Steel_Stress_Strain_Response}
\end{figure}

\begin{figure}[htbp]
	\centering
	\includegraphics[width=0.7\textwidth]{Chapter-5/figs/Diplacement_Strain_RSN1505.pdf}
	\caption{Strain hysteresis}
	\label{fig:Steel_Strain_Response}
\end{figure}

\subsection{Effect of Ground Motion Sequences}

The as-recorded ground motion sequences found in the NGA2West database \cite{Ancheta2014} did not induce an increase in the strain demands for corroded RC structures. It was observed that the strains obtained with only the mainshock and the mainshock-aftershock sequence had the same strain demands as shown in \fref{fig:ms_as_results}. Therefore, the mainshock dominated the strain demand for all the ground motions, and the subsequent aftershocks did not appear to increase the highest demand further. The results from this analysis are congruent with studies on pristine condition structures subjected to ground motion sequences\cite{Ruiz-Garcia2011}\cite{Zhai2014}. Other studies that have used the incremental dynamic analysis or scaling factors on the mainshock-aftershock sequences have increased the demands due to these sequences, and it is possible that altering the records modifies their dynamic properties and induces the observed increase in the demands\cite{Raghunandan2015}\cite{Tesfamariam2015}. At the same time, all these studies have been performed on shallow crustal earthquakes. Changing the ground motion regime to subduction earthquake sequences could have a different outcome. Only the mainshocks were used to save computational costs for the rest of this study.

\begin{figure}[htbp]
	\centering
	\includegraphics[width=0.65\textwidth]{VAC Thesis 2.0/Chapter-5/figs/MS_AS_results_noincrease_in_demands.pdf}
	\caption{Non Linear Time History Analysis (NLTHA) results showing no increase in the strain demands due to as recorded sequence of mainshock and aftershock}
	\label{fig:ms_as_results}
\end{figure}

\subsection{Effect of Corrosion Level on Strain Demands}

For all 144 columns and all 56 mainshocks, the strain demands $(\varepsilon)$ were plotted against the spectral displacement at the effective period and the equivalent structure damping ($Sd(T_{eff},\xi)$). These two parameters seem to have a linear correlation for all the corrosion levels evaluated. These results are shown in \fref{fig:all_results_nltha}. Through the post-processor subroutine, the results obtained from the NLTHA were separated for each corrosion level (CL) and the MSA methodology was used to analyze the effect of corrosion level in the strain demands.

\begin{figure}[htbp]
	\centering
	\includegraphics[width=0.65\textwidth]{VAC Thesis 2.0/Chapter-5/figs/All_results_NLTHA_Figure.pdf}
	\caption{Non Linear Time History Analysis (NLTHA) results for Strain demands vs Spectral Displacement at Effective Period $(T_{eff})$, and Equivalent Damping $(\xi)$}
	\label{fig:all_results_nltha}
\end{figure}

The multi-stripe analysis(MSA) was used to obtain a series of cumulative distribution functions to evaluate the effect of corrosion and other variables. The most impactful variables are the corrosion level and the axial load ratio. The effect of these variables are shown in \fref{fig:CDF_strain_vs_ALR}. These figures show that as corrosion level increases, the mean spectral displacement required to reach a given limit state is decreased. Similarly, the effect of the axial load ratio becomes substantial as it increases.

\begin{figure}[htbp]
	\centering
	\includegraphics[width=0.9\textwidth]{VAC Thesis 2.0/Chapter-5/figs/CDF_summary.pdf}
	\caption{Cumulative Distribution Function (CDF) for steel strain limit states and different Axial Load Ratios (ALR)}
	\label{fig:CDF_strain_vs_ALR}
\end{figure}

In order to observe the effect of corrosion level and axial load ratio, the mean spectral displacement, defined as the point where the probability of reaching a limit state is 50\% ($P(\varepsilon>\varepsilon_{ls}=0.5$), is plotted against these two variables. The results are shown in \fref{fig:mean_prob_vs_CL}.

From the results shown in \fref{fig:mean_prob_vs_CL}, two main tendencies can be observed. First, as corrosion level increases, the mean spectral displacement required to reach a limit state decreases, thus precluding an earlier limit state in a corroded structure. It also appears that there is a significant drop at a corrosion level of 10\% ($CL=10\%$), especially for the damage control limit state and the ultimate limit state. This result indicates that a corrosion level this high has potentially undesirable consequences. Research performed on RC members has shown that corrosion levels of 10\% are brittle. Thus, these results are congruent with those observations.

The second observation is that columns with an Axial Load Ratio (ALR) greater than 10\% show a more significant drop in the spectral displacement required to reach a limit state as the corrosion increases. This ALR value is commonly found in bridges, thus the importance of limiting the corrosion level that bridge columns develop in their service years.

\begin{figure}[htbp]
	\centering
	\includegraphics[width=0.85\textwidth]{VAC Thesis 2.0/Chapter-5/figs/Analysis_of_Mean_SDs.pdf}
	\caption{Analysis of mean values $(P=0.5)$ for performance limit states}
	\label{fig:mean_prob_vs_CL}
\end{figure}

The results obtained from these analyses are important since they provide limiting values for the level of corrosion that could be acceptable in new structures and assessment of existing structures. Therefore, a limiting value for the level of corrosion in new designs should be between 0\%-5\%. For existing structures, a maximum acceptable level of corrosion could be set at $CL=10\%$. This recommendation is made because existing structures would not have been designed to account for aging or corrosion inhibitors. These existing structures would have to consider the effective properties and reduce the cross-section of the reinforcing steel bars. 

Further studies on limit states of corroded structures while improving the outcomes here,  would identify and improve the design and assessment of RC members.

\subsection{Discussion of results}

The Nonlinear Time History Analysis results show that, in general, there is an increase in the strain demands due to corrosion. The increase in the strain demands depends on the limit state being evaluated and the axial load ratio. 

For all limit states, a corrosion level of 10 percent is a reasonable estimate of the maximum acceptable corrosion level, as values greater than this result in a significant change in limit state displacements. This result agrees with observed behavior in large-scale tests conducted on RC members \cite{Ma2012}\cite{Ma2018}. An ideal maximum level of corrosion is a 5\% corrosion level.

A limitation of this study was the use of limit state equations based on pristine condition elements with modern detailing. While the limit state equations provide the overall effect of corrosion in RC structures, future studies will provide more information on limit states for corroded RC columns.

Therefore, the analytical program agrees with observed experimental material tests conducted as part of this research and observations on corroded RC member tests. Thus, for the design of new structures, it is recommended that an acceptable level of corrosion range is 5\%-10\%. Moreover, for assessment, it is recommended that if a corrosion level greater than 10\% is found, an NLTHA must be performed to assess the existing structure. This analysis will have to consider the effective mechanical properties of the reinforcing steel and the reduction in the cross-sectional properties. These recommendations are applied in the following chapter. 

Another interesting finding was that the effect of recorded sequences (mainshock-aftershock) appears to be negligible. This result is in agreement with previous studies performed on MDOF systems that included degradation of the system in steel frames \cite{Ruiz-Garcia2011}. However, other studies that rely on altered records obtain the opposite result because they use amplifiers to change the peak ground acceleration (PGA), thus changing the frequency content of the records.