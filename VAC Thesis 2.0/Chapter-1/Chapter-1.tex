\chapter{Introduction}
\label{chap-one}

Structures are designed assuming their original condition remains intact through their service life. However, as structures age, they suffer various forms of degradation. In addition, they may be subjected to multiple discrete seismic events. Both of these items may impact structural performance. For example, consider an RC column and the limit state corresponding to bar buckling. The limit state displacement ($\Delta$) to achieve bar buckling for a pristine column is different from a column subjected to corrosion.
Similarly, multiple small seismic events may predispose the column to suffer longitudinal bar buckling for a lower seismic intensity level. This research aimed to incorporate the effects of aging and multiple ground motions into the definition of performance limit states. This research can potentially serve two purposes: a) for existing structures, assessment can consider likely current conditions, and b) for new structures, changes could be proposed to the initial design that could mitigate the effect of future condition degradation.

Structures subjected to multiple events and aging conditions should be considered in Performance-Based Earthquake Engineering (PBEE). Recent earthquake sequences such as the Christchurch 2010, Umbria-Marche Earthquake 1997, and more recently the Puerto Rico Earthquakes 2020, have shown that structures after sustaining damage during a mainshock have then collapsed or sustained increased damage after being subjected to a large magnitude aftershock\cite{Amato1998}\cite{Bradley}\cite{Miranda2020}. Researchers have used the Park and Ang damage index (DI) to quantify damage, expressed as a two-term expression. The first term relates to the maximum displacement, and the second term relates to the inelastic energy dissipation \cite{Young-JiPark1985}. The second term is associated with the inelastic cyclic behavior of structural components. In addition, the calibration factor is minimal and contributes little to the damage index. If the damage index renders the inelastic cyclic behavior negligible, it cannot accurately represent damage.
Further, this damage index uses calibrated data to determine the strength degradation parameter that has a degree of arbitrariness, which is undesirable \cite{Williams1995}. In addition to the Park and Ang damage index, other measures of damage such as drift ratio-based limit states have been incorporated into the PEER Performance-Based Design Probabilistic Framework\cite{Padgett2007}\cite{Ghosh2015}\cite{Shekhar2018}. The majority of these studies show an increase in the probability of damage or even collapse of a structure due to repeated loading or aging conditions, such as high corrosion levels (CL). However, these results are based on the limitations presented by these damage measures \cite{Shekhar2018}. In this study, strain limits are used as damage indicators for RC bridge columns. Specifically, concrete compressive and reinforcing steel tensile strain limits are used as damage measures \cite{Goodnight2016}. These strain limits have been correlated to observed damage in large-scale column tests. Therefore, we believe that our research will provide a realistic measure of the increase in damage for different limit states due to aging conditions and multiple earthquake loadings. 

In addition, structures can have an existing conditions such as corrosion that further deteriorates the structure's performance. Corrosion is one of the aging conditions that more significantly deteriorates the seismic response of a structure. Thus, it is crucial to determine the limit states of corroded reinforcing steel. Currently, the literature has developed expressions that correlate the level of corrosion to the decrease in strength of the reinforcing steel\cite{Yuan2017a}\cite{Du2005}. However, these studies have utilized an accelerated corrosion process that does not consider the protective film that is developed on the reinforcing steel surface when it is embedded in concrete, a process known as passivation of the reinforcing steel \cite{Mehta2014}\cite{Ghods2009}. The protective film on the reinforcing steel bar must erode to enable the corrosion process. This process is known as depassivation. Depassivation of the reinforcing steel dramatically affects the behavior of reinforcing steel and can significantly modify the measured properties of the corroded reinforcing steel.
Furthermore, no study has presented performance limit states on corroded reinforcement. Therefore, this research aims to close this gap by performing an experimental campaign. This experimental campaign consists of a series of tension tests and buckled bar tension tests to help define the performance limit states of corroded reinforcement. These results will then inform the computational model.

Moreover, there is a high likelihood for structures in a high seismic region to be subjected to a mainshock-aftershock sequence during their service life. Therefore, it is vital to consider the effects of mainshock-aftershock sequences. A series of condition-dependent nonlinear time history analyses are performed on a cantilever RC bridge column. The analysis applies a series of MS-AS sequences for different structure ages. The structure's material properties are changed due to the aging conditions (e.g., corrosion). At the end of each analysis, the main variables of the study are the limit state that was reached, the controlling mode of response (flexural or shear controlled), the equivalent viscous damping, and the accumulated deformations. This research considered the likely future condition of a structure in defining strain-based performance limit states.

\section{Scope and layout}
This document describes the main components, objectives, and results of the graduate studies of the author of this document. Chapter 2 contains the literature review, which summarizes state of the art in damage measurements and performance-based design framework. Chapter 3 covers the experimental program in corroded reinforcing steel, and Chapter 4 presents the experimental results. Chapter 5  shows the analytical program to analyze the change in the structural demands due to corrosion and discusses the results. Chapter 6 shows the application of the results from this study to new designs and assessment of existing structures that consider corrosion. Finally, Chapter 7 presents the main results of this research. 