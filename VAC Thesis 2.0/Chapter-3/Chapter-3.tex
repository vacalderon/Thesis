\chapter{Experimental program}
\label{chap-three}

Corrosion is the main aging condition that affects structures. Therefore, the success of this study relies on the most accurate representation of the corrosion process in the analytical model. As mentioned in Chapter 2, previous studies developed accelerated corrosion rebar specimens with high current densities, but did not account for the depassivation process of the reinforcing steel. A main objective of this study was to verify the mechanical properties of corroded rebars while considering the depassivation process in rebars. The experimental process was divided in to following main components, (1) corroded specimens preparation, (2) tension tests on corroded rebars, and (3) buckled bar tension (BBT) tests on corroded rebars (4) 3D scans of corroded rebars, (5) SEM observations of fracture surfaces from BBT tests, and (6) tension and BBT tests on turned down corroded rebars. The results obtained from the tension and BBT tests were be used in the analytical model presented in Chapter \ref{chap-five}, and more broadly provide accurate mechanical properties of corroded rebars.

\section{Specimen preparation}

Rebars embedded in concrete generate a protective film that protects the rebars against corrosion, due to the alkaline environment of the cement paste. The most common form of corrosion occurs via chloride attacks. During a chloride attack, the chloride diffuses in the concrete cover and enters in contact with the surface of the rebars. This initiates the process of depassivation in which the protective film on the surface of the rebar is eliminated. The depassivation of the rebar enables the process of corrosion to occur. The proposed experimental campaign described in this chapter aims to simulate the process of corrosion as it would occur in rebars embedded in concrete. Using as received rebars, that is no special treatment to the surface of the rebar, the specimen preparation procedure consist of (1) protect the rebar ends that will be used as grips during the tension and perform buckled bar tension (BBT) tests, (2)  generate the passive layer on the rebars, (3) perform accelerated corrosion of the rebars until the specified levels of corrosion.

\subsection{Preparation of rebar ends}

First, it was necessary to protect the areas of the rebar that are going to function as grips for the rebars during the tension and BBT tests against corrosion,to prevent a failure inside the grips. Therefore, the ends of the rebars were protected with three layers. The first layer consisted of two-part epoxy, the second layer consisted of electroplater tape, and the third layer consisted of shrink tube. These layers ensured minimal corrosion in the grip areas of the rebar. \fref{fig:RebarSpecimenGeomtry} and \fref{fig:RebarEndsProtection} show the specimen geometry and the protective layers at the ends of the rebars.
\begin{figure}[htbp]
	\centering
	\includegraphics[width=1.0\textwidth]{Chapter-3/figs/RebarSamples}
	\caption{Rebar Specimen Geometry}
	\label{fig:RebarSpecimenGeomtry}
\end{figure}

\begin{figure}[htbp]
	\centering
	\includegraphics[width=0.7\textwidth]{Chapter-3/figs/Rebar_Ends}
	\caption{Rebars Ends Protection}
	\label{fig:RebarEndsProtection}
\end{figure}
\newpage 

To optimize the space and time it takes to prepare the rebars for corrosion, a large parent rebar that contains three specimens was prepared. This ensured that the parent specimen, placed in a corrosion cell, provided the same level of corrosion to all the specimen subsets. After each parent specimen had the specified level of corrosion, the large specimen was cut into the three smaller specimens shown in \fref{fig:RebarSpecimenGeomtry}.

\begin{figure}[htbp]
	\centering
	\includegraphics[width=1.0\textwidth]{Chapter-3/figs/LargeSpecimen}
	\caption{Large specimen containing three subset of specimens}
	\label{fig:LargeSpecimen}
\end{figure}

\newpage

\subsection{Passivation of the rebars}

In order to simulate the conditions of rebars embedded in concrete, it was necessary to generate the passive layer on the surface of the rebars. There are two ways to generate the passive layer: (1) embed rebars in concrete and wait for the passive layer to generate, or (2) submerge the reinforcing steel in a synthetic pore solution that mimics the cement paste environment. The second option is more suited for material testing since it does not involve demolishing the concrete. To this regard, Ghods et al \cite{Ghods2010} developed ten different pore solutions to generate the passive film on the surface of rebars. The pore solutions that were developed in their study intended to mimic the cement paste. Their study conducted 10 solutions designed to encompass concentrations of $Ca^{+2}$, $Na^{+}$, $K^{+}$, and $(SO_{4})^{+}$ found in the cement paste. The solution that generated the passive layer with the best quality of protective film, based on passive current density, was used in this study. To achieve the desired concentration of the $Ca^{+2}$, $Na^{+}$, $K^{+}$, and $(SO_{4})^{+}$ ions, the following concentration of chemicals recommended in \cite{Ghods2010} was used in the pore solution:

\begin{itemize}
	\item Saturated calcium hydroxide $Ca(OH)_2$ (approx. 1.7 g/L)
	\item Sodium hydroxide $Na(OH)$ (4.00 g/l)
	\item Potassium hydroxide $(KOH)$ (11.22 g/l)
	\item Calcium sulfate dehydrate $Ca(SO)_4 + 2H_2O$ (13.77 g/l)
\end{itemize}

The rebars were placed in the pore solution as received without any special surface preparation in the area of study, since any form of special surface preparation affects the quality of the passive layer \cite{Andersson1989}, \cite{DawnMarcotte2001}, \cite{Moragues1987}, and \cite{Page1983}. In addition, Ghods et al showed that no significant change in the anodic polarization of the rebar surface after being placed in the pore solution for 8 days or 14 days. Their results are shown in \fref{fig:GhodsRebarPassivation}. Therefore the rebars were placed in the pore solution for a minimum of 8 days. 

\begin{figure}[htbp]
	\centering
	\includegraphics[width=0.6\textwidth]{Chapter-3/figs/AsReceived_AnodicPolarization_time}
	\caption{As received rebars anodic polarization immersed in pore solution for different times\cite{Ghods2009}}
	\label{fig:GhodsRebarPassivation}
\end{figure}

To prepare the rebars for the development of the passive layer, a specimen preparation assembly was developed. The assembly consisted of placing the rebar inside a PVC pipe, close the ends of the pipe  with a $90^{\circ}$ PVC elbow, and closing the open ends with PVC pipe plugs. The assembly is shown in \fref{fig:RebarPassivation}. This assembly ensured that the pipe remained airtight and prevented the carbonation of the calcium hydroxide ($Ca(OH)_{2}$ in the pore solution.

\begin{figure}[htbp]
	\centering
	\includegraphics[width=1.0\textwidth]{Chapter-3/figs/AnodicPolarization_01}
	\caption{Assembly for rebar preparation for the development of passive layer in pore solution}
	\label{fig:RebarPassivation}
\end{figure}

\newpage
\subsection{Accelerated corrosion  of reinforcing steel}

Four components are required for corrosion to occur: (1) the anode, (2) the cathode, (3) an electrolytic connection, and (4) an electrical path. The corrosion cell deployed in this study used this concept in the following way: the anode consisted of the rebar, the cathode consisted of a stainless steel mesh, the electrolytic connection was made with a sodium chloride solution, and the electrical path was forced via an electric circuit. The corrosion cell uses the assembly developed to generate the passive layer with the components needed for corrosion as shown in \fref{fig:AcceleratedCorrosion}. The circuit was designed so that the current in the specimen was $150mA$, equivalent to $340\mu A/cm^2$ on the bare surface of the rebar, to the author's knowledge this is the lowest current density ever used on an accelerated corrosion process for corroded rebars subjected to tension and BBT tests. Since the rebar and stainless steel mesh have a low resistance, a $100\Omega$ resistor will be added to the circuit, such that it stabilizes the current and keeps it constant at $150mA$.

\begin{figure}[htbp]
	\centering
	\includegraphics[width=0.95\textwidth]{Chapter-3/figs/AcceleratedCorrosionProcedure}
	\caption{Accelerated corrosion process}
	\label{fig:AcceleratedCorrosion}
\end{figure}

Ghods et al \cite{Ghods2010} determined that for rebars with passive films, a concentration of 0.3 Moles of sodium chloride ($NaCl$) will start the depassivation process on the rebars. Accordingly the same sodium chloride solution was used in this study. To estimate the time to apply the current and obtain the desired level of corrosion,  Faraday's law shown in equation \ref{eq.FaradayEq} was used.

\begin{equation}
	m_{loss}=\frac{it(AM)}{nF}
	\label{eq.FaradayEq}
\end{equation}

 In equation \ref{eq.FaradayEq}, $m_{loss}$ corresponds to the mass loss, $i$ is the current in amperes ($i=5 mA$), $t$ is the time the current is sustained in seconds $(AM)$ is the atomic mass of the oxidizing component. For this study the oxidizing component is the iron ($Fe$) in the rebars, hence $(AM)=54.845g/mol$, $n$ is the number of electrons lost per atom oxidized, for $Fe$ the number of electrons is equal to 2, and $F$ is Faraday's number ($F=96485 C$). Solving \eref{eq.FaradayEq} for $t$ and assuming uniform corrosion for different corrosion levels, the time of application was calculated for the total gauge length in the parent rebar specimen. The estimated times to achieve the corrosion levels are shown in Table \ref{tab:AcceleratedCorrosionTime}. 

\begin{table}[htbp]
	\caption{Accelerated corrosion times in 3/4" rebar total gauge length in parent specimen}
	\label{tab:AcceleratedCorrosionTime}
	\centering	
		\begin{tabular}{l c c}
		Corrosion Level (CL) & Mass loss (g)   & time (days)     \\  \hline	
		5\%                  & 57.6            & 16    \\	
		10\%                 & 115.3            & 31     \\	
		15\%                 & 172.9            & 47   \\	
		20\%                 & 230.5            & 63     \\
		25\%                 & 288.1               & 78   \\
		\end{tabular}
\end{table}

\section{Tension tests}

The tension tests were performed to evaluate differences in the stress-strain behavior of corroded reinforcing steel with those found in the literature. In addition, the data obtained from these tests served as an input to the analytical model. The tension tests were performed in accordance with the standard ASTM A370, which specifies the loading procedure and the gauge lengths. 
\subsection{Tension test procedure}

The tension tests consist of:
\begin{enumerate}
    \item Place the rebar in the universal testing machine
    \item Pull the rebar in tension 
    \item Record the load and the strain on the rebar 
\end{enumerate}

The strains were captured through the use of LED markers from the Optrotrak Certus HD system. The gauge length between the LED markers will be 2inches as is specified in the ASTM A370 standard, \fref{fig:TensionTest} shows an example of the test setup. The stress will be calculated based on the load reading from the UTM machine and divided by the measured area of the corroded rebars.

\begin{figure}[htbp]
	\centering
	\includegraphics[width=0.4\textwidth]{Chapter-3/figs/TensionTest}
	\caption{Tension test setup example\cite{Overby2016}}
	\label{fig:TensionTest}
\end{figure}

\subsection{Testing parameters}
The results obtained from the tension tests were valuable to understand the change in the stress-strain relationship, the yield strength, and ultimate strength of corroded rebars. These values are obtained as outlined in ASTM A370, and are summarized below.


\textbf{Yield strength}: Yield strength is the stress at which a material exhibits a specified limiting deviation from the proportionality of stress to strain. To determine the yield strength the offset method was used to obtain this value

\textbf{Ultimate strength}: Ultimate strength is calculated by dividing the maximum load the specimen sustains during a tension test by the original cross-sectional area of the specimen.

\textbf{Uniform Axial Elongation}: The elongation is the increase in length of the gauge length, expressed as a percentage of the original gauge length. In recording elongation values, give both the percentage increase and the original gauge length.

Regression analysis of the results helped update that equations that correlate these parameters to the corrosion level, and potentially yield an updated form of  \eref{eq.eleven}. The outcomes of the tension tests will further improve the results of the analytical model in Chapter \ref{chap-five}.

\section{Buckled bar tension (BBT) tests}

One of the limit states considered in the seismic design of bridge columns is the fracture of buckled rebar, which corresponds to the ultimate limit state. Recent tests have been developed to determine the critical bending strain of buckled reinforcing steel \cite{Barcley2019}, which is associated with the sudden fragile fracture of buckled longitudinal reinforcement of RC column. The buckled bar tension (BBT) test simulates the bending and tension strain demands on a buckled bar to determine critical bending strain in buckled rebars. 

\subsection{Test procedure}
Barcley et al \cite{Barcley2019} developed a methodology to calculate local strains on a buckled bar using an LED optical sensor system \cite{NorthernDigitalInc.2020}. \fref{fig:BBTseq} shows the basic concept of the BBT test.

\begin{figure}[htbp]
	\centering
	\includegraphics[width=0.7\textwidth]{Chapter-3/figs/BBT_Sequence}
	\caption{BBT Test sequence\cite{Barcley2019}}
	\label{fig:BBTseq}
\end{figure}

The procedure to perform the buckled bar tension test consisted of:

\begin{enumerate}
    \item First, the rebar was placed in the universal testing machine (UTM), and the corroded rebar specimen was prepared with LED markers on the surface, such that the displaced shape of the bar could be measured.
    \item Second, the rebar specimen was compressed to impose a bending strain of a prescribed level. Barcley et al showed that a fourth order polynomial can be fit to the LED sensors near the buckled region of the bar to obtain the displaced shape ($w$). The bending strain was calculated using solid mechanics principles. The curvature is the second derivative of the displaced shape ($w$) for small displacements, which is calculated as: 
    \begin{equation}
        \phi=\frac{\frac{d^2w(x)}{dx^2}}{\left[1+\left(\frac{dw(x)}{dx}\right)\right]^\frac{3}{2}}\approx \frac{d^2w(x)}{dx^2}
        \label{eq.CuvatureAprox}
    \end{equation}
    If we assume that the bending is symmetric for the rebar then the strain in the extreme fibers of the rebar is calculated as:
    \begin{equation}
        \varepsilon_{b}=\phi\left(\frac{d_{bl}}{2}\right) 
        \label{eq.BendingStrain}
    \end{equation}    
    Combining equations \eref{eq.CuvatureAprox}, and \eref{eq.BendingStrain}, the bending strain can be expressed as:
    \begin{equation}
        \varepsilon_{b}=\frac{d^2w(x)}{dx^2}\left(\frac{d_{bl}}{2}\right) 
        \label{eq.BendingStrainExpanded}
    \end{equation}
    An example of the calculation of the bending strain is shown in \fref{fig:BBT_Curvature}
    \item Third, once buckled to the prescribed curvature, the bar is loaded in tension until fracture is observed
    \item Then the process was repeated with a different bar for a different bending strain. After all the tests were performed results from elongation at peak force can be generated as the example shown in \fref{fig:BBT_MaxBendingStrain}. From the results obtained through BBT tests the critical bending strain was determined. The critical bending strain is defined as the point at which a low elongation under load is obtained. This low elongation results in a brittle fracture of the rebar as shown in \fref{fig:BBT_DuctileBrittle}(b). \fref{fig:BBT_MaxBendingStrain} shows that for bars with rebars the critical bending strain is $\varepsilon_{b}=0.10$ for grade 80 ksi steel.
\end{enumerate}

\begin{figure}[htbp]
    \centering
    \includegraphics[width=0.8\textwidth]{Chapter-3/figs/BBT_Curvature}
    \caption{a) Picture of buckled bar; (b) Position of optical markers and adjustment to neutral axis; (c) Calculation of curvature \cite{Barcley2018}}
    \label{fig:BBT_Curvature}
\end{figure}
\begin{figure}
    \centering
    \includegraphics[width=1.0\textwidth]{Chapter-3/figs/BBT_Ductile_vs_Brittle}
    \caption{(a)Ductile rebar fracture; (b) Brittle rebar fracture \cite{Barcley2018}}
    \label{fig:BBT_DuctileBrittle}
\end{figure}
\begin{figure}[htbp]
    \centering
    \includegraphics[width=1.0\textwidth]{Chapter-3/figs/BBT_MaxBendignStrain}
    \caption{BBT results for rebars with and without ribs \cite{Barcley2018}}
    \label{fig:BBT_MaxBendingStrain}
\end{figure}

\newpage
\subsection{Test Matrix}
The BBT tests are proposed for different levels of corrosion such that any changes on the behavior are studied and incorporated in the analytical model. The proposed test matrix is shown in Table \ref{tab:Test Matrix}. It must be noted that in each corrosion level for the BBT tests, each experiment will correspond to a different prescribed bending strain. The selection for each prescribed bending strains consist of 1) Start with the maximum bending strain determined in previous research to be around $\varepsilon=0.10$ as shown in \fref{fig:BBT_MaxBendingStrain}, 2)if brittle fracture is observed the next test will be at a lower bending strain for example $\varepsilon=0.08$, 3)otherwise higher strains such as $\varepsilon=0.12$ will be evaluated. This is repeated for each corrosion level. It is expected that six tests per corrosion level will be sufficient however, more tests will be performed if necessary. The results obtained will be compared to those of a pristine condition rebar which corresponds to a corrosion level of $CL=0\%$

% Please add the following required packages to your document preamble:
% \usepackage{multirow}
% \usepackage[table,xcdraw]{xcolor}
% If you use beamer only pass "xcolor=table" option, i.e. \documentclass[xcolor=table]{beamer}
\begin{table}[htb]
	\caption{Corroded Rebar Test Matrix}
	\label{tab:Test Matrix}
	\centering	
	\begin{tabular}{|c|c|c|c|}
	\hline
	\multicolumn{4}{|c|}{\cellcolor[HTML]{CC0000}{\color[HTML]{FFFFFF} Corroded rebar test matrix}}                                               \\ \hline
	\multicolumn{1}{|l|}{Test}     & \multicolumn{1}{l|}{Diameter of bar} & \multicolumn{1}{l|}{CL (\%)} & \multicolumn{1}{l|}{Number of Tests} \\ \hline
	                               &                                      & 0                            & 3                                    \\ \cline{3-4} 
	                               &                                      & 5                            & 3                                    \\ \cline{3-4} 
	                               &                                      & 10                           & 3                                    \\ \cline{3-4} 
	                               &                                      & 15                           & 3                                    \\ \cline{3-4} 
	                               &                                      & 20                           & 3                                    \\ \cline{3-4} 
	\multirow{-6}{*}{Tension test} & \multirow{-6}{*}{\#6}                & 25                           & 3                                    \\ \hline
	                               &                                      & 0                            & 6                                    \\ \cline{3-4} 
	                               &                                      & 5                            & 6                                    \\ \cline{3-4} 
	                               &                                      & 10                           & 6                                    \\ \cline{3-4} 
	                               &                                      & 15                           & 6                                    \\ \cline{3-4} 
	                               &                                      & 20                           & 6                                    \\ \cline{3-4} 
	\multirow{-6}{*}{BBT test}     & \multirow{-6}{*}{\#6}                & 25                           & 6                                    \\ \hline
	\end{tabular}
\end{table}

\subsection{Testing parameters}
\textbf{Maximum bending strain}

\newpage

\section{3D Scanning}

The FARO Arm \cite{FAROTechnologiesInc.2022} was used to perform 3D scans of the gauge lengths on the corroded specimens. The procedure consists of using a probe that sends a laser signal to the surface of the specimens, the reflections from these laser are captured by the probe and the data is acquired in a computer for later post-processing. The probe used in this research had an accuracy of 0.018 mm. The system can be seen in fig X.X.

%insert figure of faro arm here

The main use of this tool was to obtain an accurate reading of the diameter of the rebar, which will provide with an accurate reading of the tension and BBT tests results. The reader is referred to Chapter \ref{chap-four} to see the results obtained from this methodology. The data collected in this study can be used in future research to study the effect of imperfections using finite element analysis methods.

\section{SEM observations}

A total of 6 fracture surfaces were obtained from the BBT tests. Three fracture surfaces were from ductile failure and the remaining three fracture surfaces were obtained from brittle failures. The collected fraacture surfaces were observed under the the Scanning Electron Microscope (SEM). In this study a Variable Pressure SEM (VPSEM) was used since it is more suited to study metals.

The SEM was used with two main objectives. These objectives were: (1) determine if there is any change in the microstructural composition of the fracture surface, and (2) using the Energy Dispersive Spectroscopy (EDS) of the VPSEM to determine the presence of chlorides or oxides in the fracture surface. 

\subsection{SEM imaging}
As explained previously SEM imaging can provide with valuable information regarding the type of fracture that was observed at the microstructural level and determine any change in the micro structure of the fracture surface.

\subsection{EDS Spectrum Analysis}

EDS consists in the use of back scatter electrons (BE) to estimate the elements that are present in a given SEM observation and relies the interaction of  source of X-ray excitation and the sample. This technology uses the fundamental principle that each element has a unique atomic structure allowing a unique set of peaks on its electromagnetic emission spectrum. From this analysis, a spectrum of the elements found for each of the observed the samples was obtained. 

\section{Turned down corroded rebars tests}

The bars that did not reach their intended corrosion level were repurposed to be turned down. The objective of doing this was to verify that indeed the virgin material of the specimens does not change. But rather the results observed in tension tests and BBT tests on the corroded rebars correspond to a geometrical effect of the imperfections caused by corrosion on the surface of the specimens. To prove this a set of three tension tests and six BBT tests were performed on rebars that were turned down to remove all the imperfections caused by the corrosion process.

\subsection{Test matrix}

The test matrix shown below summarizes the rebars used for these tests. The table shows the final turned down diameter, and the type of test for each repurposed specimen.

\subsection{Testing parameters}
The testing parameters corresponded to the same studied in the tension and BBT tests performed on the corroded rebars: (1) Yield Strength, (2) Ultimate Strength, (3) Uniform elongation, and (4) maximum bending strain.

\section{Expected outcomes from experimental phase}

The results from the tension tests will help to establish if the depassivation process in the corroded rebars has an effect on the measured stress-strain relationship of steel as has been observed in previous research that did not considered the depassivation process \cite{Meda2014},\cite{Yuan2017a},\cite{Du2005}. Similarly the results from the BBT tests will show any changes in the critical bending strain of rebars as they corrode. The critical bending train impacts the bar fracture limit state. In the case of corroded rebars it is expected that the critical bending strain will be modified by changes in the mechanical properties of steel due to corrosion, and effects of concentrated corrosion will be visible along the surface of the rebars. We hypothesize that these two factors will induce fracture at a lower bending strain than those observed in pristine conditions rebars\cite{Barcley2019}.

The results obtained from the experiment will also be used to define the bar fracture limit state for RC columns. Research currently being developed at NC State will provide models that allow us to establish bar fracture tensile strain while considering different parameters such as transverse steel spacing, axial load ratio, and strength of the concrete to mention a few. The model will be similar to that presented by Barcley et al \cite{Barcley2018} shown in \eref{eq.BarFracture}. 
\begin{equation}
    \varepsilon_{t}=\frac{ln(\frac{\varepsilon_{b}}{0.001})}{\frac{300p}{f'c A_{g}}+\frac{0.7}{\rho_{t}}}
    \label{eq.BarFracture}
\end{equation}
This model could then be implemented in the analytical model presented in Chapter \ref{chap-four}. If fracture occurs at a lower strain in corroded rebars, this implies that the corroded RC columns are prone to reach the fracture bar limit state at a much lower displacement than in a pristine RC column.

Future studies will verify the application of the results found in this research to perform full scale corroded RC column that consider the effect of depassivation in the cyclic behavior of such columns.