\chapter{Condition dependent performance based design and Assessment}

\label{chap-five}

Developing a design procedure that provides resilient bridges is one of the main goals of this study.This research will develop a displacement based design tool, that achieves that goal. This chapter firts introduces the overall design concept; then the direct displacement based design (DDBD) is reviewed and proposed modifications to this design methodology, based on corroded RC physical tests results, are shown. Finally a schedule of work is presented.

\section{Recommendations to new designs from material tests}
\section{Design of new structures considering aging conditions}
\subsection{Existing methodologies}

\textbf{Life 365}
Is a software developed by the Life 365 Consortium III to estimate the service life and life-cycle costs of alternative concrete mixture designs aand corrosion protection systems. The softwaqre is able to incorporate local data such as Average temperature throughout the year, Corrosion concentration limits, type of environment (Tidal zone, Spray zone, 800m from the ocean, 1.2 km from the ocean), the type of structure (Parking garage, Urban highway, Rural highway). One of the assumptions is that the deterioration time after initiation of corrosion is constant at 6 years.

\textbf{FIB 34 Model Code}
fib Bulletin 34 addresses Service Life Design (SLD) for plain concrete, reinforced concrete and pre-stressed concrete structures, with a special focus on design provisions for managing the adverse effects of degradation. Its objective is to identify agreed durability related models and to prepare the framework for standardization of performance based design approaches.

Four different options for SLD are given:
\begin{enumerate}
    \item Full probabilistic approach,
    \item Semi probabilistic approach (partial factor design),
    \item Deemed to satisfy rules
    \item Avoidance of deterioration.
\end{enumerate}




\textbf{AASHTO R19A}

This code is similar to FIB34. The main premise is to determine the different exposures areas of a bridge, and on the basis of the exposures the designer quantifies based on mix design the probability that corrosion will occur. This is a full probabilistic approach in which the goal is to prevent corrosion with a 10\% probability that it will occur.


\subsection{Proposed methodology}

While the methodologies available are good in allowing the structural engineer to evaluate different mix options and preventing corrosion to occur or to delay it and propose a repair and maintenance during the life of a structure.  However, this tools do not provide a way to estimate the degradation of the structural performance of a structure in the case corrosion does occur. It might also not be economically feasible to provide technologies such as stainless steel reinforcement. Therefore it might be beneficial for stake holders to evaluate what will occur if they do prescribe a better concrete mix that allows them to have a delay on corrosion occurrence, and to know what the performance level will be given a maximum level of corrosion that can occur.

Design the mix and materials to meet desired performance levels at different locations of the structure (AASHTO R19A, FIB 34, Life365)
If the constraints applied permit corrosion evaluate maximum corrosion level for expected age of the structure
Obtain corrosion level from reduced bar diameter
With corrosion level modify the  effective material properties of the reinforcing steel. 
Obtain limit states for corresponding corrosion level or the effective properties of the corroded reinforcing steel
Evaluate structure for displacement response spectrum at the location of the structure
Verify if the structure reaches a given performance level at the maximum estimated corrosion level
If the structure does not reach the collapse prevention and it does not exceed the permitted ductility level the design is successful
If the structure reaches the damage control performance or is close the designer can choose a different bar diameter or change the mix, material to be used in the structure.
\textbf{Quantifying corrosion}
After time to corrosion is achieved the diameter of the bar will reduce with:
\begin{equation}
    d_{b}(t)=d_{bi}-\lambda(t)
\end{equation}

$\lambda(t)$ is the rate of corrosion, while there are many equations that have tried to determine the rate of corrosion in terms of the water to cement ratio and the cover. A simpler approach that is widely accepted is rate of 0.5 mills per year. Therefore the bar diameter change can be expressed as:
\begin{equation}
    d_{b}(t)=d_{bi}-0.0005(t-T_{corr})
\end{equation}

Therefore the corrosion level can be expressed as:
\begin{equation}
    CL=1-\left(\frac{d_{b}(t)}{d_{bi}}\right)^2=1-\left(\frac{d_{bi}-0.0005(t-T_{corr})}{d_{bi}}\right)^2
\end{equation}

Here is an outline on the design procedure for a new structure considering the steps presented in the previous slide.

Incorporates current life service processes available to engineers
It provides a deterministic approach on what is the performance of the structure as it reaches a given level of corrosion.
The user can specify the maximum allowable level of corrosion in this case it is set to 10%
Presents enough flexibility to add different corrosion rate models from simple to more complex
For this approach to work it is necessary to compliment the design with  good Q/C and Maintenance plans



\subsection{Application example: Bridge 547 (AKDOT)}

\section{Assessment of existing structures considering aging conditions}

\subsection{Existing methodologies}

\subsubsection{ASCE 41}

ASCE 41 

\subsubsection{SLaMA}

Simple Lateral Mechanism Analysis (SLaMA) is apart of the Siesmic Assessement of Existing Buildings Guidelines from the New Zealand Society for Earthquake Engineering (NZCEE) \cite{NZSEE2019},  Displacement-based assessment (DBA) procedure that focuses on establishing the probable displacement capacity of the primary lateral system. DBA utilizes displacement spectra which can more readily and directly represent the response of a building in earthquake shaking. Displacement-based methods use the same methods as a force-based assessment to determine the force-displacement response of the structure. However, the expected displacement demand is based on the structural characteristics (effective stiffness and equivalent viscous damping) at the assessed displacements rather than on initial elastic characteristics. Displacement spectra set for different levels of elastic damping or ductility are used rather than the acceleration spectra reduced for ductility used for force-based design. The displacement-based approach enables degrading strength and the influence of poor
hysteretic response characteristics to be incorporated in the analysis. Similarly, the concepts can be extended to seismic retrofit design. The basic steps for the SLaMA methodology are enumerated below:

\begin{enumerate}
    \item Assess the structural configuration and load paths to identify key structural elements, potential structural weaknesses (SWs) and severe structural weaknesses (SSWs).
    \item Calculate the relevant probable strength and deformation capacities for the individual members.
    \item Determine probable inelastic behaviour of elements by comparing probable member capacities and evaluating the hierarchy of strength.
    \item Assess the sub-system inelastic mechanisms by extending local to global behaviour.
    \item Form a view of the potential governing mechanism for the global building by combining the various individual mechanisms and calculate the probable base shear and global displacement capacitymeasured at the top of the primary lateral structure. The global displacement capacity will typically be limited to that for the system with the lowest displacement capacity.
    \item Determine equivalent SDOF system, seismic demand and \%NBS.
\end{enumerate}

The \%NBS is used as an indicator of risk and defines a structural weakness within the context of SLaMA. SW is defined as an aspect of the building structure and/or the foundation soils that scores less than 100\%NBS. Note that an aspect of the building structure scoring less than 100\%NBS but greater than or equal to 67\%NBS is still considered to be a structural weakness even though it is considered to represent an acceptable risk. An example of the evaluation of the \%NBS is shown in figure.

While this method uses a displacement based assessment methodology, and therefore a better estimate on the performance of the structure is possible, the methodology does not explicitly address how to consider corrosion or aging in the assessment of the structure.

\subsection{Proposed Condition Dependent Performance Based Seismic Assessment}

In order to provide tools for practicing engineers to assess aging structures, and specifically include corrosion in their assessment of existing structures, a methodology that uses the Displacement Based Assessment is proposed. Similar to the case of new structures design the degradation of the material properties are included. Some assumptions are necessary to estimate the time of initiation of corrosion of the structures, therefore for corroded RC structures it is necessary to:

\begin{enumerate}
    \item Obtain information from construction: as built drawings, material certifications, mix design, observations during construction
    \item Historical data: Inspection reports, observed deterioration
    \item Site measurements: Corrosion rate (using GalvaPulse)
    \item If it is not possible estimate corrosion level from measured diameter or the rate of corrosion, empirical equations relating the water to cement ratio ($w/c$), bar diameter ($d_{b}$) and cover ($c$) are available in the literature or by using life 365.
    \item Estimate probable mechanical properties of materials
    \item Apply DBA procedure
\end{enumerate}

A 

\subsection{Application Example: Bridge 547 (AKDOT)}